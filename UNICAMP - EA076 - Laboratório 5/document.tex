\documentclass[12pt, a4paper]{article}
%=========================== PACKAGES =============================%

\usepackage[utf8]{inputenc}
\DeclareUnicodeCharacter{00A0}{ }

\usepackage[hmargin=1.5cm,vmargin=1.5cm]{geometry}
\usepackage[brazil]{babel}

\usepackage{longtable}

\usepackage{graphicx}
\usepackage{placeins}
\usepackage{subcaption}
\usepackage{float}

\usepackage{hhline}
\usepackage{courier}
 
\usepackage{amsmath}
\usepackage{bm}
\usepackage{amsfonts}

\usepackage{hyperref}

\usepackage{listings}
\renewcommand\lstlistingname{Programa}
 

\usepackage{color} %red, green, blue, yellow, cyan, magenta, black, white
\definecolor{mygreen}{RGB}{28,140,0} % color values Red, Green, Blue
\definecolor{mylilas}{RGB}{170,55,241} 
\lstset{language=Matlab,%
    %basicstyle=\color{red},
    basicstyle=\footnotesize\ttfamily,
    breaklines=true,%
    morekeywords={matlab2tikz},
%    keywordstyle=\color{blue},%
    morekeywords=[2]{1}, keywordstyle=[2]{\color{black}},
%    identifierstyle=\color{black},%
%    stringstyle=\color{mylilas},
%    commentstyle=\color{mygreen},%
    showstringspaces=false,%without this there will be a symbol in the places where there is a space
    numbers=left,%
    numberstyle={\tiny \color{black}},% size of the numbers
    numbersep=9pt, % this defines how far the numbers are from the text
    %emph=[1]{for,end,break},emphstyle=[1]\color{red}, %some words to emphasise
    %emph=[2]{word1,word2}, emphstyle=[2]{style},    
    frame= single,
}
 
\lstdefinestyle{nonumbers}
{numbers=none}

\usepackage{multirow}

\usepackage{wrapfig}
\usepackage{float}

%=========================== PACKAGES =============================%


\author{Gustavo Ciotto Pinton}

\begin{document}

\begin{titlepage}
\vspace*{.28\textheight}
\begin{center}
%
\begin{figure}[h]
    \centering
    \includegraphics[scale=0.18]{image/LogoUnicamp}
\end{figure} 
%
\vspace*{10pt}
%\text{ }\\[7 cm]
\textbf{\LARGE Experimento 5 - Conversor Digital Analógico} \\ \vspace{12pt}
\textbf{\large EA076 - Laboratório de Sistemas Embarcados}
\vspace*{72pt}

Gustavo \textbf{CIOTTO PINTON} - \textbf{RA 117136}

Anderson \textbf{UNE BASTOS} - \textbf{RA 093392}

\vspace{30pt} 

Grupo \textbf{6} - Turma \textbf{C}
 
\vspace{36pt}
Campinas, \today

\end{center}
\end{titlepage}

\newpage
%  
% {\large 
%     \centerline{\textbf{Exercício de Fixação de Conceitos 1}}
%     \centerline{Gustavo Ciotto Pinton - 117136}
%     \centerline{EA072 - Inteligência Artificial em Aplicações Industriais}
% }
\section {Introdução}

O objetivo deste experimento foi sintetizar diversas formas de onda através do
conversor digital-analógico disponível do \textit{kit} KL25z. Foi possível,
portanto, gerarmos ondas quadradas, triangulares e senoidais, através de ferramentas
matemáticas, como a série de Taylor, cujo propósito é fornecer aproximações para
tais funções.

\vspace{12pt}

Para a última parte do experimento, desenvolvemos uma aplicação envolvendo o LCD
que fornece diversas opções de interatividade ao usuário.

\vspace{12pt}

As próximas seções são dedicadas ao detalhamento das implementações realizadas. 

\section {Análise}

Esta seção é dedicada às características que determinam as três montagens.

\subsection{Criação do Projeto} 


\section* {Referências bibliográficas}
\begin {itemize}
  \item \url{https://en.wikipedia.org/wiki/Sunspot}. Acessado às 19:22 29/09/2015.
  \item Guyon, I.; Elisseeff, A. "An introduction to variable and feature selection", Journal of Machine Learning Resear ch, vol. 3, pp. 1157 - 1182 2003.
\end{itemize}

    
\end{document}
