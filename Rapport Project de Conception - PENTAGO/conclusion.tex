\section {Conclusion}

In order to program an artificial intelligence able to play Pentago, we have
employed Matlab language, an alpha-beta pruning algorithm (first-depth) and a
regular laptop computer. The results obtained reach our expectations:

\begin{enumerate}

	\item Our graphical interface performed well. Easy to use, functional and
	enjoyable.

	\item Our program may enable matches between humans, between AIs and mixed.

	\item The processing time demanded by each play is reasonable.

	\item The AI is efficient. It is roughly comparable in strength to the one
	developed by our advisor, which was set in the beginning of the project as our
	parameter of quality.

\end{enumerate}

Nevertheless, an AI capable of strongly solving Pentago has been already
programed with C++ and runned on a supercomputer at NERSC. Therefore, it is
clear that may identify some room for improvement:

\begin{enumerate}
  	\item Programming Language: The use of C++ language instead of Matlab would
  	enable better access to and better performance of the computational
  	resources. For instance, it allows the algorithm used at NERSC to code and
  	manipulate data in a very ingenious way, which enables a better solution.

	\item Algorithm: From embodying transition tables to speed up the performance
	to increasing the depth of the alpha-beta search, there are some possible
	improvements. Some of them, for instance the increase of the depth of search,
	are not very well suited for an ordinary computer as it implies a greater
	overall processing time. The strong solving algorithm used at NERSC succeeded
	by associating the use of in game symmetries with brute force, all that being
	allowed by the way used to store data.

	\item Hardware: NERSC's supercomputer, Edison, has « a peak performance of
	2.57 petaflops/sec, 133,824 compute cores, 357 terabytes of memory, and 7.56
	petabytes of disk » (Nersc, 2015). Meanwhile, our algorithm is developed to
	perform in laptops. It may be difficult to have access to a Cray XC30
	supercomputer as Edison, but there cheaper and more accessible options on the
	market. For instance, one may exploit a regular computer GPU using CUDA to
	rise performances (NVIDEA,2015).

\end{enumerate}

Lastly, those improvements might be hard to achieve for the time and the effort
that they demand. They are maybe more suited for a longer and deeper project
that surpasses a final graduation project. Furthermore, the results obtained on
this project may be used by future students who want to solve Pentago or other
board games on Matlab. They might also use our work, as well as our advisor AI,
to compare different Pentago AIs. This last approach may allow some interesting
results, for instance it may be used as a tool for evaluating non deterministic
AIs.

\newpage
