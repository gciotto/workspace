\section{Introduction}

\subsection{The Objective}

This work focus on the implementation of an artificial intelligence able to
play effectively the board game Pentago. The programming language that will be
deployed to achieve so is Matlab. Most Pentago AIs are developed in C++, hence
using Matlab would bring some innovation for there is little documentation on
programming a Pentago AI with this language. Moreover, this choice has a
pedagogical value: we have previously worked with Matlab and our advisor has
done a Pentago AI of his own before, in such way that it would be easier to
compare results and approaches for the evaluation of our work.

\vspace{10pt}

For simplicity reasons we will employ << he >> instead of << he or she >> on
the writing.

\subsection{What is Pentago?}

<< \textit{Pentago} is a two-player abstract strategy game invented by Tomas
Flodén. The Swedish company Mindtwister has the rights of developing and
commercializing the product.

\vspace{10pt}

The game is played on a 6x6 board divided into four 3x3 sub-boards (or
quadrants). Taking turns, the two players place a marble of their color (either
black or white) onto an unoccupied space on the board, and then rotate one of
the sub-boards by 90 degrees either clockwise or anti-clockwise. A player wins
by getting five of their marbles in a vertical, horizontal or diagonal row
(either before or after the sub-board rotation in their move). If all 36 spaces
on the board are occupied without a row of five being formed then the game is a
draw.>> (Pentago, 2015).


\subsection{Why Pentago?}

When Pentago was introduced in 2005, it immediately attracted the interest of
Computer Science community, as it presents itself as <<[...] a two player,
deterministic, perfect knowledge, zero sum game: there is no random or hidden
state, and the goal of the two players is to make the other player lose (or at
least tie).>> (Pentago is a first player win, 2015). Games that hold those
characteristics, as Chess and Go, are of great interest for the computer
science and frequently of great popularity worldwide.

\vspace{10pt}

<< But where does the motivation to examine games comes from? Herik et al. [43]
described, the interest of Artificial Intelligence researchers in strong
game-playing programs as an important goal for more than half a century now.
”The principal aim is to witness the ”intelligence“ of computers. A second aim
has been to establish the game-theoretic value of a game, i.e., the outcome
when all participants play optimally.“ Heule et al. [19] added the motivational
question: ”Can artificial intelligence outperform the human masters in the
game? “. Yet it is important to mention, that the methods to show intelligence
and outperform human master can differ substantially from those for solving.
But altogether, games seem to be an interesting site to show Artificial
Intelligence. Moreover solving games applies the Artificial Intelligence
methods to larger problems which may become computational puzzles. Even so the
topic of solving may be seen as a toy application, it is perfect to show that
large knowledge based problems are solvable.>> (On solving Pentago, 2015).

\vspace{10pt}

By solving we mean the computational sense of the word: finding an optimal
strategy that would either proof that a player (the first or the second to
play) always win or that both may force a draw. The strength of a solution may
be categorised in as stated in (On solving Pentago, 2015):


\begin{enumerate}
  \item Ultra-weak: It is known what is the result of the game when it is
  perfectly played, although the strategy that leads to this result is not known.
	
  \item Weak: A strategy that leads from the initial position(s) to the
  perfect-play result for both players is known.
  
  \item  Strong: For all possible to have positions, a strategy that leads to
  the perfect-play result is known.

\end{enumerate}

\vspace{10pt}

We may observe that «strongly solve» implies «weakly solved» that implies «ultra-weakly solved». 

\vspace{10pt}

Unfortunately, many games present such complexibility to evaluate the possible
outcomes that it takes too much time to trace a strategy, hence rendering the
solution unreachable with the contemporane technologie. As it presents some
symmetries and <<not so many>> states, Pentago is more easily solved than the
previous examples of Chess and Go.

\vspace{10pt}

<< The 6x6 version of Pentago has been strongly solved with the help of a Cray
supercomputer at NERSC. With symmetries removed, there are
3,009,081,623,421,558 possible positions. If both sides play perfectly, the
first player to move will always win the game.>> (Pentago is a first player win,
2015).

\vspace{10pt}

All in all, we may conclude that the solution of Pentago belongs to a class of
problems (solving games with an AI) that has been subject of everlasting
interest of the scientific community and general public. It is an already
solved problem, but it has been solved employing a supercomputer. Moreover,
there is little documentation on programming Pentago with Matlab instead of C++.

\vspace{10pt}

We want to produce with this work an efficient AI, employing Matlab language
and conventional computers to run it. Finally, we want it to be fast, as it
should be suited for tests against other available AI and human players, in
such fashion that we want to make it play in less than a couple of minutes.

\subsection{Work Planning}

We will deploy the following phases of our project:

\begin{enumerate}
  \item First, we will have a player vs player enabled Pentago.
  		\begin{itemize}
  		  \item Creation of a graphical interface that enables players to play a
  		  piece on an empty space when it is his turn.

  		  \item Modification of this interface to allow players to rotate the board
  		  after their play.
  		  
  		  \item Introduction of a <<game over>> clause, that ends the game as a
  		  player wins or there is a tie.
  		\end{itemize}
  		
  	\item Second, we will introduce an artificial intelligence able to play
  	against an external player (i.e. player vs AI enabled).
  		\begin{itemize}
  		  \item Insertion of a random play AI able to play against a player, i.e.
  		  obeying the rules of the game.
  		  \item Research about possible solving algorithms to implement.
  		\end{itemize}
  		 
  	\item Finally, we will work on the optimization of our AI, rendering its
  	algorithm more efficient and robust (i.e. strong AI).
  		\begin{itemize}
  		  \item Implementation of efficient solving algorithms.
  		  \item Further tuning for optimal results.  
  		\end{itemize}
\end{enumerate}

In order to advance from a module of work to the next one, we will employ
unitary and integration tests for the validation of the progress. Likewise, we
will employ tests and simulations to verify the performance of our choices in
the tuning phase of our algorithm, for instance confronting our artificial
intelligence with the one programmed by our advisor.

\newpage
