\documentclass[12pt]{article}

\usepackage{sbc-template}

\usepackage{graphicx,url}
\usepackage{csvsimple}

\usepackage[brazil]{babel}    
\usepackage[utf8]{inputenc}
 
\sloppy

\title{Contagem do número de instruções dos \textit{benchmarks} do
\texttt{SPEC CPU2006} e implementação de uma \textit{pintool}}

\author{Gustavo Ciotto Pinton\inst{1} }


\address{Instituto de Computação -- Universidade Estadual de Campinas
(UNICAMP)\\
  Av. Albert Einstein, 1251, Cidade Universitária, Campinas/SP \\
  Brasil, CEP 13083-852 \\  Fone: [19] 3521-5838
  \email{ra117136@unicamp.br}
}

\begin{document} 

\maketitle

\begin{abstract}
This report describes the number of instructions executed by each
benchmark in \texttt{SPEC CPU2006}. It was achieved thanks to the
Intel's \texttt{pin} application. All results were obtained by running the
benchmarks with the \texttt{ref} inputs and a single iteration. A new
pin tool was equally developed, which lists the number of executed instructions
of every routine of all threads that compose the program. Five benchmarks,
run with inputs belonging to the \texttt{test} set, were used to test this new
tool.
\end{abstract}
     
\begin{resumo} 
Este relatório apresenta o número de instruções executadas por cada
\textit{benchmark} presente no \texttt{SPEC CPU2006}, calculado através da
utilização da ferramenta \texttt{pin}, desenvolvida pela Intel. Os valores
encontrados correspondem às entradas do tipo \texttt{ref} e a uma única
iteração. Além disso, desenvolveu-se uma nova \texttt{pin tool}, cuja saída
é a uma lista com o número de instruções executas pelas rotinas de um programa,
separadas pela sua respectiva \textit{thread}. Tal ferramenta foi testada em 5
benchmarks na configuração \texttt{test}.
\end{resumo}


\section{Introdução}

\texttt{SPEC}, do inglês \textit{Standard Performance Evaluation Corporation}, é
uma organização constituída por fabricantes de \textit{hardware},
\textit{software} e instuições de pesquisa, cujo objetivo é definir uma série de
testes relevantes e padronizados para a análise da performance de um computador.
Tais testes podem ser igualmente denominados de \textit{benchmarks} e visam
avalisar um aspecto específico de processamento. Durante a aula, vimos que
existe uma grande variedade de \textit{benchmarks} disponíveis na
\textit{internet}, sendo que alguns foram implementados, por exemplo, para
avaliar o desempenho de aplicações \textit{web}. Um \textit{benchmark} realiza
um conjunto de operações definidos, chamado de \textbf{workload}, e produz um
resultado, ou seja, uma \textbf{métrica} que tenta avaliar o desempenho do
computador submetido ao respectivo \textit{workload}. Tendo em vista tais
conceitos, o \texttt{SPEC CPU2006} é um conjunto de 31 \textit{benchmarks} que
procuram avaliar a performace de três componentes principais, sendo eles o
processador, a arquitetura de memória e os compiladores. Os \textit{benchmarks}
são distribuídos em dois \textit{suites}, denominados de \texttt{CINT2006} e
\texttt{CFP2006}, que se distinguem quanto à natureza do seu processamento
intensivo: o primeiro é focado na performance das operações que utilizam números
\textbf{inteiros}, sendo que o segundo, de números em \textbf{ponto flutuante}.
Essencialmente, \texttt{SPEC CPU2006} oferece duas métricas, \textbf{speed} e
\textbf{rate} (ou \textbf{throughput}), medindo, respectivamente, o quão rápido
um computador completa uma única tarefa e quantas tarefas tal sistema pode
realizar em um pequeno período de tempo. Cada métrica possui, por sua vez,
quatro \textit{variações}, que se diferenciam quanto ao \textit{suite} (inteiro ou
ponto flutuante) e ao método de compilação. Em relação a este último, duas
opções são disponibilizadas: \texttt{base}, que apresenta requisições mais
estritas, isto é, as \textit{flags} de compilação devem ser usadas na mesma
ordem para todos os \textit{benchmarks} de uma dada linguagem e é exigida para
um teste \textit{reportable} (execução que pode ser publicada), e \texttt{peak}, opcional
e com menos exigências.

\texttt{PIN}, por sua vez, é uma ferramenta para instrumentação e análise de
programas, à medida que ela permite a inserção de código dinamicamente ao
executável. Esta ferramenta é capaz de interceptar a execução da primeira
instrução e gerar um novo código a partir dela. Uma \texttt{pintool} pode ser
definida como uma extensão do processo de geração de código realizada pelo
\texttt{pin}, já que é capaz de interagir com este último e comunicar quais
funções, ou \textit{callbacks}, o aplicativo deve inserir ao código. De maneira
geral, uma \texttt{pintool} é composta por dois componentes, chamados de
\textit{instrumentation code} e \textit{analysis code}. O primeiro deve decidir
onde inserir o novo código, isto é, em que locais as rotinas de análise deverão
ser lançadas. É nessa fase, portanto, que características \textbf{estáticas} do
código, tais como, por exemplo nome de rotinas ou número de instruções que as
compõem, devem ser exploradas. O segundo, por sua vez, é chamado à medida que o
código é executado e, dessa forma, pode afetar significamente a performance de
um executável se o determinado código apresentar complexidade elevada. 

Neste relatório, será abordada, na primeira parte, o uso de uma \texttt{pintool}
para a avaliação do número de instruções executadas por cada um dos
\textit{benchmarks} e, em seguida, a implementação de uma nova ferramenta capaz
de determinar quantas instruções cada rotina de cada \textit{thread} foram
executadas.

\section{Contagem das instruções} \label{sec:count}


A fim de calcular as instruções de cada \textit{benchmark} do \texttt{SPEC
CPU2006}, dois \textit{scripts bash} foram implementados. O primeiro, chamado de
\texttt{run-pintool-all-benchmarks.sh} verifica e executa o \textit{runspec},
escrito em \textit{perl}, para cada um dos \textit{benchmarks}, além de compilar
a \textit{pintool} que será utilizada. Tal \textit{script} comunica alguns
parâmetros importantes ao comando \textit{runspec}, tais como o método de
compilação (\texttt{base}), o conjunto de entradas que será transmitido aos
programas (\texttt{ref}, no nossa caso), o número de iterações (1) e,
evidentemente, o nome do \textit{benchmark}. A execução deste comando produz um
arquivo de extensão \texttt{.tmp.log} no diretório do projeto, que é copiado
posteriormente a uma pasta, cujo nome é igual ao do
\textit{benchmark} que acabou de ser executado. O arquivo de configuração
transmitido ao comando \texttt{runspec} faz referência ao segundo \textit{script},
\texttt{run-spec-command.sh}, responsável por relacionar o aplicativo
\texttt{pin} com o respectivo \textit{benchmark}. Este \textit{script} recebe
como parâmetro o comando que o \texttt{SPEC} utiliza e o retransmite para o
\texttt{pin}, que é responsável por executá-lo efetivamente.

No manual de referência do \texttt{pin}, são indicadas quatro maneiras distintas
de se calcular o número de instruções executadas por um programa. A primeira,
\texttt{inscount0}, insere a rotina de análise antes de cada instrução,
produzindo, assim, uma grande perda de performance. A segunda,
\texttt{inscount1}, é superior à anterior, à medida que utiliza uma outra medida
de granularidade, chamado de \texttt{BBL} (do inglês, \textit{basic block}) e,
portanto, economiza diversas chamadas à função de análise. A terceira rotina,
chamada de \texttt{inscount2}, usa o mesmo princípio que a anterior, porém
apresenta melhor desempenho, visto que faz uso de dois recursos a mais que
\texttt{inscount1}. O primeiro recurso é a mudança de \texttt{IPOINT\_BEFORE}
para \texttt{IPOINT\_ANYWHERE}, que autoriza o \texttt{pin} escolher em que
ordem a função de análise é colocada, permitindo, assim, que ele escolha o ponto
que requeira mínimas operações de salvamento e restaturação dos registradores.
Além disso, esta ferramenta também usa a opção de \textit{fast call linkage},
que explora o fato de que alguns compiladores podem eliminar o
\textit{overhead} que, para funções pequenas como a a função de análise, é
comparável ao próprio conjunto de operações da respectiva função. Esta opção é
ativada através do uso de \texttt{PIN\_FAST\_ANALYSIS\_CALL}. Por fim, o último
programa utiliza, além dos recursos comentados anteriormente, uma unidade de
armazenamento rápido, chamado de \texttt{TLS}, indexado pelos \textit{indexes}
das \textit{threads} para gerar o número de instruções por \textit{thread}. 

Tendo em mente estas características, escolhe-se o programa
\textbf{\texttt{inscount2}} para o cálculo do número de instruções, visto que
este último apresenta o maior número de otimizações e que, neste contexto, não
visa-se encontrar uma contagem por \textit{threads}, mas sim global. Os
\textit{benchmarks} foram rodados em apenas um \textit{core} do processador
\textit{Intel i3} e levaram, ao todo, 20 horas aproximadamente. Os resultados
encontrados estão listados abaixo. Alguns \textit{benchmarks} possuem mais de
uma entrada e, portanto, produzem mais de uma saída. Tais saídas estão separadas
por \textit{vírgulas} na lista abaixo:

Para os \textit{benchmarks} do conjunto \textbf{inteiro}:
\begin {itemize}
\item \texttt{400.perlbench}: 1109198045367, 384794966182, 707786630157
\item \texttt{401.bzip2}: 432459815466, 181180973754, 309587458388,
553737482083, 605707320979, 345617094417
\item \texttt{403.gcc}: 77489490228, 151321617974, 139512528163, 103537789803,
113969504551, 154631785799, 183463430994, 169999746216, 58461954145
\item \texttt{429.mcf}: 341546845898
\item \texttt{445.gobmk}: 234525674455, 625631596347, 325147399858, 235593319559, 339210226918
\item \texttt{456.hmmer}: 971544939130, 2052353062674
\item \texttt{458.sjeng}: 2309967978778
\item \texttt{462.libquantum}: 2291912513237
\item \texttt{464.h264ref}: 500835844419, 355915755256, 3188039031373
\item \texttt{471.omnetpp}: 576122145483
\item \texttt{473.astar}: 411719281496, 829803428219
\item \texttt{483.xalancbmk}: 1048497777434
\end {itemize}

Para os \textit{benchmarks} do conjunto em \textbf{ponto flutuante}:

\begin {itemize}
\item \texttt{410.bwaves}: 2.495.514.310.671
\item \texttt{416.gamess}: 1124505753634, 878108843916, 3766238100560
\item \texttt{433.milc}: 1175358805504
\item \texttt{434.zeusmp}: 2016616007502
\item \texttt{435.gromacs}: 1971200720706
\item \texttt{436.cactusADM}: 2655006820074
\item \texttt{437.leslie3d}: 4626149683423
\item \texttt{444.namd}: 2361163844939
\item \texttt{447.dealII}: 1903231296730
\item \texttt{450.soplex}: 377431323949, 389903616072
\item \texttt{453.povray}: 1002529177253
\item \texttt{454.calculix}: 6894341894243
\item \texttt{459.GemsFDTD}: 2722715227347
\item \texttt{465.tonto}: 3563812458171
\item \texttt{470.lbm}: 1314569317678
\item \texttt{481.wrf}: 3867428098913
\item \texttt{482.sphinx3}: 3432419178361
\item \texttt{998.specrand}: 536611748
\item \texttt{999.specrand}: 536611748
\end{itemize}

Verifica-se, portanto, que o \textit{benchmark} que utiliza mais entradas em
seus testes é \texttt{403.gcc}, totalizando 1.152.387.847.873 instruções
executadas. Os \textit{benchmarks} que rodaram mais e menos instruções foram,
respectivamente, \texttt{454.calculix} e
\texttt{998.specrand}/\texttt{999.specrand}.


\section{Implementação de uma nova \textit{pin tool}} \label{sec:tool}

Tendo em vista que os programas apresentados na seção anterior não permitem a
contagem de instruções por rotina e por \textit{thread}, propõe-se a
implementação de uma nova \textit{pintool} contendo estas duas funcionalidades.
Para isso, utiliza-se a API da ferramenta para a adição de funções de
\textit{instrumentação} para cada rotina e de \textit{análise} para suas
instruções. Além disso, faz-se uso do \textit{TLS} para a armazenagem de
informações específicas a uma \textit{thread}.

O registro da função de instrumentação é realizada através de uma chamada a
\texttt{RTN\_AddInstrumentFunction()}, cujo  parâmetro é uma
\textit{callback function} que é executada uma vez para cada rotina do
programa. Dentro desta função, processa-se todo o comportamento estático do
programa, isto é, explora-se o seu estado independente da sua execução. Cada
rotina é descrita por uma estrutura de dados, cujo nome é \texttt{RTN}. Tal
estrutura permite que diversas informações sejam extraídas, como por exemplo, o
nome da rotina, através de \texttt{RTN\_Name()}, o número de instruções
estáticas dentro dela, com \texttt{RTN\_NumIns()} e quais instruções pertencem a
ela. Este último é realizado através da iteração de uma lista ligada, acessada
através de \texttt{RTN\_InsHead()} ou \texttt{RTN\_InsTail()}. No entanto, tais
funções devem ser chamadas somente depois de \texttt{RTN\_Open()} ser executada,
conforme documentado no manual da ferramenta.

O programa define duas novas classes, \texttt{Thread\_node} e
\texttt{Routine\_node}, responsáveis por armazenar informações sobre as
\textit{threads} e rotinas executadas. Os objetos instanciados destas classes
compõem duas listas ligadas, sendo que cada nó do tipo \texttt{Routine\_node}
contém o início e o fim de uma lista de objetos \texttt{Thread\_node}. Tal
escolha de implementar uma lista de \texttt{Thread\_node} dentro de cada
instância de \texttt{Routine\_node}, e não o inverso, foi adotada a fim de
melhorar a performance da rotina de análise, que será explicada mais adiante.
Além de dois ponteiros para o início e fim de uma lista ligada de
\texttt{Thread\_node}s, um objeto da classe \texttt{Routine\_node} possui o nome
da rotina, obtido através de \texttt{RTN\_Name()}, a referência para o próximo
nó da lista e o \textit{id}, que é dado pela função \texttt{RTN\_Id()}.
\texttt{Pin} atribui a cada rotina uma identificação única globalmente, isto é, 
mesmo se uma determinada rotina com mesmo nome aparecer em duas imagens distintas,
as duas \textit{cópias} terão \textit{ids} diferentes. Um objeto da classe
\texttt{Thread\_node} possui, por sua vez, além de um ponteiro para o próximo
elemento da lista, o \textit{id} da \textit{thread}, que pode ser recuperado
através do \texttt{THREAD\_ID} e o número de instruções que a rotina daquela
respectiva \textit{thread} executou.

A criação de nós do tipo \texttt{Routine\_node} é realizada na rotina de
instrumentação por motivos ligados à performance, visto que é um processo
extremamente custoso. Tal afirmação reside no fato de que uma rotina
pode aparecer mais de uma vez em \textit{threads} distintas e que, por este
motivo, é necessário percorrer a lista toda vez a fim de procurá-la. Observa-se
também que um programa pode apresentar um número elevado de rotinas - os
\textit{benchmarks} testados, por exemplo, apresentam mais de 1000 delas. Dessa
forma, caso tais operações fossem inseridas na função de análise, a performance
do programa seria afetada muito negativamente. Sendo assim, após a verificação
da existência da rotina na lista e a possível criação em caso negativo,
inicia-se o processo de posicionamento das funções de análise. Cada estrutura do
tipo \texttt{RTN} possui uma referência para uma lista das instruções que a
compõem, sendo acessada pela função \texttt{RTN\_InsHead()} e percorrida por
\texttt{RTN\_Next()}. Para cada instrução, executa-se, portanto, a função
\texttt{INS\_InsertCall()} para inserir a \textit{callback} de análise. Essa
função pode receber, além da referência da \textit{callback}, a ordem em que
ela é chamada (antes ou depois, por exemplo, da execução da instrução) e um
conjunto variável de parâmetros. \texttt{PIN} oferece uma série de
possibilidades para a passagem de parâmetros à função de análise. No nosso caso,
por exemplo, dois argumentos são utilizados, sendo eles o \textit{id} da
\textit{thread}, que é passado através de \texttt{THREAD\_ID}, definido na
enumeração \texttt{IARG\_TYPE}, e o ponteiro do objeto do tipo
\texttt{Routine\_node}, que identifica a qual rotina a instrução pertence. Este
último exige a adição de dois parâmetros a \texttt{INS\_InsertCall()}: um para a
especificação do tipo do argumento, isto é, \texttt{IARG\_PTR}, e o segundo o
ponteiro propriamente dito. Para a ordem de execução, quatro opções são
disponíveis, sendo elas \texttt{IPOINT\_BEFORE}, \texttt{IPOINT\_AFTER} ,
\texttt{IPOINT\_ANYWHERE} e \texttt{IPOINT\_TAKEN\_BRANCH}, equivalendo,
respectivamente, à execução da rotina de análise antes, depois, em um ponto
determinado pelo próprio \texttt{PIN} ou se um \textit{branch} foi executado.
Como não há nenhuma preferência de ordem e prefere-se aquela que ofereça melhor
desempenho, escolhe-se \texttt{IPOINT\_ANYWHERE}.

Por último, a criação de nós do tipo \texttt{Thread\_node} é realizada na função
de análise juntamente com o incremento do número de instruções executadas. É 
necessário, porém, a fim de não comprometer siginificamente a execução do
programa, implementarmos uma função que execute o mínimo de operações. Neste
contexto, acessa-se a lista de \texttt{Thread\_node}s através do ponteiro do
tipo \texttt{Routine\_node}, passado como parâmetro, e procura-se o nó, cujo
\textit{id} seja igual ao segundo argumento da função. Caso ele não seja
encontrado, um novo nó é criado. Vale lembrar, portanto, que o desempenho do
programa é fortemente prejudicado caso ele possua muitas \textit{threads}. Isso
porque a lista ligada de \texttt{Thread\_node}s é sempre percorrida a cada nova
instrução. Por fim, o contador de instruções é incrementado.

Esta \texttt{pintool} foi testada em cinco \textit{benchmarks} do
\texttt{SPEC CPU2006}, sendo eles \texttt{400.perlbench}, \texttt{401.bzip2},
\texttt{403.gcc}, \texttt{429.mcf}  e \texttt{445.gobmk}. Todos eles apresentam
somente uma \textit{thread}. A seguir, são apresentadas algumas considerações
importantes a respeito das execuções.

\section{Conclusões}

% https://www.spec.org/cpu2006/Docs/readme1st.html#Q1

\bibliographystyle{sbc}
\bibliography{sbc-template}

\end{document}