\documentclass[12pt]{article}

\usepackage{sbc-template}

\usepackage{graphicx,url}

\usepackage[brazil]{babel}   
\usepackage[utf8]{inputenc}
 
\sloppy

\title{Contagem do número de instruções dos \textit{benchmarks} do
\texttt{SPEC CPU2006} e implementação de uma \textit{pintool}}

\author{Gustavo Ciotto Pinton\inst{1} }


\address{Instituto de Computação -- Universidade Estadual de Campinas
(UNICAMP)\\
  Av. Albert Einstein, 1251, Cidade Universitária, Campinas/SP \\
  Brasil, CEP 13083-852 \\  Fone: [19] 3521-5838
  \email{ra117136@unicamp.br}
}

\begin{document} 

\maketitle

\begin{abstract}
This report describes the number of instructions executed by each
benchmark in \texttt{SPEC cpu2006}. It was achieved thanks to the
Intel's \texttt{pin} application. All results were obtained by running the
benchmarks with the \texttt{ref} inputs and a single iteration. A new
pin tool was equally developed, which lists the number of executed instructions
of every routine of all threads that compose the program. Five benchmarks,
run with inputs belonging to the \texttt{test} set, were used to test this new
tool.
\end{abstract}
     
\begin{resumo} 
Este relatório apresenta o número de instruções executadas por cada
\textit{benchmark} presente no \texttt{SPEC cpu2006}, calculado através da
utilização da ferramenta \texttt{pin}, desenvolvida pela Intel. Os valores
encontrados correspondem às entradas do tipo \texttt{ref} e a uma única
iteração. Além disso, desenvolveu-se uma nova \texttt{pin tool}, cuja saída
é a uma lista com o número de instruções executas pelas rotinas de um programa,
separadas pela sua respectiva \textit{thread}. Tal ferramenta foi testada em 5
benchmarks na configuração \texttt{test}.
\end{resumo}


\section{Introdução}


\section{Contagem das instruções} \label{sec:count}


\section{Implementação de uma nova \textit{pin tool}} \label{sec:tool}


\section{Conclusões}


\bibliographystyle{sbc}
\bibliography{sbc-template}

\end{document}