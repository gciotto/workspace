\documentclass[12pt, a4paper]{article}


%=========================== PACKAGES =============================%
\usepackage[utf8]{inputenc}
\usepackage[hmargin=1.1cm,vmargin=1.1cm]{geometry}

%--- LANGUE--%
\usepackage[brazil]{babel}
\usepackage{hyperref}
%--- LANGUE--%

%--- FONT ----%
% \usepackage{helvet}
% \renewcommand{\familydefault}{\sfdefault}
%--- FONT ----%
%=========================== PACKAGES =============================%

\begin{document}

\pagestyle{empty} 

%======== PRENON ET NOM =========%
Gustavo CIOTTO PINTON
%======== PRENON ET NOM =========%

%======== ADRESSE ===============%
Rua Voluntário Amador Lourenço, 57

Valinhos, São Paulo, Brasil CEP 13271-393

+55 19 99898-5633

 \url{gustavociotto@gmail.com}

\url{http://gciotto.github.io/workspace/}
%======== ADRESSE ===============%
\begin{flushright}

%======== ENTREPRISE ===============%

Instituto de Computação - UNICAMP

Av. Albert Einstein, 1251

Cidade Universitária, Campinas/SP, Brasil

%======== ENTREPRISE ===============%


\end{flushright}

%======== VILLE ET DATE =========%

\vspace{12pt}

Campinas,  \today
%======== VILLE ET DATE =========%

\vspace{12pt}

\textbf{Assunto:} Inscrição ao processo seletivo de aluno especial do Instituto
de Computação da UNICAMP

\vspace{12pt} 

Prezada Senhora, prezado Senhor, 

\vspace{12pt}
%======== LETTRE ===============%

Atualmente no 11º semestre do curso de Engenharia de Computação na Universidade
Estadual de Campinas, escrevo esta carta com o intuito de apresentar minhas
principais motivações e razões para cursar as disciplinas \textit{MO601B -
Arquitetura de Computadores II} e \textit{MO403B - Implementação de Linguagens
I} na situação de aluno especial no segundo semestre de 2016. Além de adquirir
mais experiência e conhecimento sobre as áreas de sistemas embarcados e
arquitetura de computadores, pretendo iniciar um mestrado no início de 2017 nas
mesmas áreas. Ainda neste contexto, tenho grande interesse no
desenvolvimento do \textit{kernel} do \textit{Linux}.

\vspace{12pt}

Sempre fui comprometido aos resultados e à qualidade dos projetos requisitados
nas diversas disciplinas durante o período de graduação. Graças a isso, possuo,
atualmente, o segundo maior CR de toda a turma e, mais importante, já tive a
oportunidade de participar de um programa de duplo diploma entre a UNICAMP e a
\textit{École Supérieure d'Électricité}, situada nos arredores de Paris. Além de
desenvolver minhas habilidades de comunicação tanto em inglês como em francês,
este intercâmbio me ajudou a clarificar meu projeto profissional, à medida que,
apesar de ser um escola predominantemente de engenharia elétrica, entrei em contato com a
programação de sistemas embarcados e a implementação de circuitos digitais.
Ainda no contexto acadêmico, fui monitor de duas disciplinas de graduação:
\textit{MC102 - Algoritmos e Programação de Computadores} e \textit{EA871 -
Laboratório de Programação Básica de Sistemas Digitais}. Tais experiências foram
muito relevantes, pois, durante o processo de ensinar, também adquiri 
conhecimento através da troca de experiências.

\vspace{12pt}

Atualmente, estou realizando um estágio no Laboratório Nacional de Luz Sincroton
(LNLS - CNPEM) no grupo de Controle, que é responsável por prover soluções de
controle e sensoriamento de diversos equipamentos utilizados no acelerador de
partículas, como fontes de tensão e corrente, bombas de vácuo e sensores de
temperatura, por exemplo.  Durante minhas atividades neste grupo, entrei em
contato com o desenvolvimento de \textit{software} e \textit{hardware} para
sistemas embarcados, especialmente para Linux embarcado, já que o laboratório
pretende utilizar a \textit{BeagleBone Black} para futuras implementações. Entre
os principais projetos, pude desenvolver módulos para o \textit{kernel} do Linux
(na ocasião, desenvolvi um sistema para determinar o \textit{delay} e a precisão com
que pulsos eram transmitidos a partir de uma rede \textit{Ethernet}), programas
para o módulo \textit{real-time} da Beagle e integrar um receptor GPS a esta
mesma placa, utilizando soluções já existentes como o \textit{GPSd}. Além disso,
tive a oportunidade igualmente de desenvolver para o microcontrolador
\textit{STM32F746} da \textit{STM32}. Neste caso, projetei um cliente que faz
requisições às diversas placas a fim de testar seu bom funcionamento.

\vspace{12pt}

Obrigado pelo tempo e esforço ao revisar minha aplicação. Aguardo ansiosamente a sua resposta.

\vspace{12pt}

Atenciosamente,

\begin{flushright}
\textbf{Gustavo CIOTTO PINTON}
\end{flushright}
%======== LETTRE ===============%


\end{document}
