\documentclass[10pt, a4paper]{article}

%=========================== PACKAGES =============================%
\usepackage[utf8]{inputenc}
\usepackage[hmargin=1.0cm,vmargin=1.0cm]{geometry}

%--- LANGUE--%
\usepackage[brazilian]{babel}
%--- LANGUE--%

%--- FONT ----%
\usepackage[scaled]{helvet}
\renewcommand{\familydefault}{\sfdefault}
%--- FONT ----%

%--- PHOTO ----%
\usepackage{graphicx}
\usepackage{wrapfig}
%--- PHOTO ----%


{\renewcommand{\arraystretch}{1.3}}

\usepackage{titlesec} % Used to customize the \section command
\titleformat{\section}{\large \bf \uppercase}{}{0pt}{}[\titlerule] % Text formatting of sections
\titlespacing{\section}{17pt}{10pt}{5pt} % Spacing around sections

%=========================== PACKAGES =============================%


\author{Gustavo Ciotto Pinton}

\begin{document}

\pagestyle{empty} 

%--- PUT YOUR PHOTO, IF NECESSARY --%
\begin{wrapfigure}{r}{3cm}
  \vspace{-20pt}
  \begin{center}
  % \includegraphics[width=3cm, height=4cm]{23}
  \end{center}
\end{wrapfigure}
%--- PHOTO --%

%======== PRENON ET NOM =========%
\textbf{\LARGE Gustavo CIOTTO PINTON}
%======== PRENON ET NOM =========%

%======== ADRESSE ===============%
Rua Voluntário Amador Lourenço, 57, Nova Itália, Valinhos

\textit{CEP 13271-393}, São Paulo, Brasil

(+55) 19 99898 5633, 22 anos

\textit{gustavociotto@gmail.com} 

\textit{fr.linkedin.com/in/gustavociotto}
%======== ADRESSE ===============%

% {\centerline {\large Elève ingénieur de deuxième année en double diplôme à
% Supélec}} 
% {\centerline {Recherche d'un stage de six mois}}


%======== FORMATION ===============%
\section{Formação Acadêmica}

\begin{tabular}{p{.16\textwidth} p{.78\textwidth}}

 \textbf{2015 - Presente} & \textbf{UNICAMP Universidade de Campinas}, Campinas,
 Brasil.
 \\
 \textbf{2011 - 2013} & Estudante de Engenharia de Computação. \vspace{8pt}\\

 \textbf{2013 - 2015} & \textbf{SUPÉLEC  École Supérieure d'Électricité},
 Gif-Sur-Yvette, França. \\
 & Uma das grandes escolas francesas de engenharia. Duplo diploma em
 engenharia elétrica e de computação. \vspace{8pt}
 \\
  
 
 \textbf{2008 - 2011} & \textbf{COTUCA Colégio Técnico de Campinas}, Campinas,
 Brasil. \\
 & Escola técnica da UNICAMP. Diploma técnico em Informática. \\
\end{tabular}

%======== FORMATION ===============%



%======== COMPETENCES ===============%
\section{Competências Linguísticas}

\begin{tabular}{p{.16\textwidth} p{.78\textwidth}}

\textbf{Português} & Língua materna.  \\ 

\textbf{Francês} & \textbf{562} pontos do \textbf{\textit{TCF}},
equivalente ao nível \textbf{C1} do CEFR. Data do teste \textit{Maio
2015}.\\

\textbf{Inglês} & \textbf{633} pontos do total de 677 do \textbf{\textit{TOEFL
ITP}}. Data do teste \textit{Março 2015}. \\  

\end{tabular}


\section{Competências Informáticas} 

\begin{tabular}{p{.16\textwidth} p{.78\textwidth}}

 \textbf{Sistemas} & Linux (Ubuntu e ArchLinux), Windows.  \\ 

 \textbf{Linguagens} & Java, C/C++, PHP,  SQL, \LaTeX, noções de Objective-C.
 \\
 
 \textbf{Software} & Eclipse, Matlab, MySQL Workbench,  Microsoft SQL Server
 (2008/2010), Microsoft Visual Studio (2008/2010),  Microsoft Office Software
 Suites (2007/2010), noções de funcionamento do Apple’s XCode. \\
\end{tabular}

%======== COMPETENCES ===============%

%======== EXPERIENCES ===============%
\section{Experiência Profissional}

\begin{tabular}{p{.16\textwidth} p{.78\textwidth}}

\textbf{08/2015 - Presente}  & Assistente de laboratório, \textbf{FEEC
 UNICAMP - Faculdade de Engenharia Elétrica e de Computação, Campinas, Brasil.}
 \\
  & \vspace{-12pt}
  \begin{itemize}
    \item Assistência aos alunos nas práticas de laboratório da
    disciplina \textit{EA871 - Laboratório de Programação Básica de Sistemas
    Digitais}, envolvendo a programação dos recursos de uma placa de arquitetura
    ARM.
    
	\end{itemize}\\

 \textbf{07/2014 - 09/2014}   & Estagiário em  produção,
 \textbf{Elis -  usina de Brétigny-sur-Orge, França.}\\
  & \vspace{-12pt}
  \begin{itemize}
    \item Preparação dos pedidos dos clientes. \vspace{-8pt}
    \item Ordenação dos produtos e operação das máquinas.
  \end{itemize} \\

 \textbf{02/2013 - 07/2013}   & Assistente de laboratório, \textbf{IC
 UNICAMP - Instituto de Computação, Campinas, Brasil.} \\
  & \vspace{-12pt}
  \begin{itemize}
    \item Assistência aos alunos de primeiro ano nas práticas de laboratório da
    disciplina \textit{MC102 - Algoritmos e Programação de Computadores}.
    \vspace{-8pt}
    \item Elaboração e concepção de ferramentas pedagógicas em C
    abordando alguns assuntos fundamentais tais como recursividade, ponteiros e
    algoritmos de procura e ordenação.
	\end{itemize}\\

 \textbf{08/2011 - 06/2012} & Assistente de pesquisa, \textbf{IC UNICAMP -
 Instituto de Computação, Campinas, Brasil}. \\
 & \vspace{-12pt}
 \begin{itemize}
 	\item Análise da performance de softwares distribuídos em toda a rede em
 	função des sistemas operacionais, da eficiência da Java Virtual Machine (JVM)
 	e das condições de hardware.
 	\end{itemize}
 \\
 
 
 \textbf{12/2010 - 02/2011} & Estagiário na área de TI - \textbf{MOVILE,
 Campinas, Brasil.}
 \\  & Responsável pela manutenção de: \vspace{-8pt}
 \begin{itemize}
   \item Banco de dados (MySQL Server). \vspace{-8pt}
   \item Sistemas web (PHP et JavaScript) da empresa.
 \end{itemize} 
\end{tabular}

%======== EXPERIENCES ===============%

%======== ACTIVITES ===============%
\section{Atividades Diversas}

\begin{tabular}{p{.16\textwidth} p{.77\textwidth}}

\textbf{Sport} & Tênis e futebol. \\ 

\end{tabular}

%======== ACTIVITES ===============%

\end{document}
