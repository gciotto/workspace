\documentclass[10pt]{beamer}
\setbeamertemplate{caption}[numbered]
\usepackage[utf8]{inputenc}
\usepackage[french]{babel}
\usepackage{amsmath}
\usepackage{graphicx}
\usepackage{stmaryrd}
\usepackage{bbm}
\usepackage{listings}
\usepackage[section]{placeins}
\usepackage{caption}
\usepackage{url}

\usetheme{Malmoe}

\usepackage{color} %red, green, blue, yellow, cyan, magenta, black, white
\definecolor{mygreen}{RGB}{28,140,0} % color values Red, Green, Blue
\definecolor{mylilas}{RGB}{170,55,241}

\lstset{language=Matlab,%
%basicstyle=\color{red},
basicstyle= \ttfamily \footnotesize,
breaklines=true,%
morekeywords={matlab2tikz},
keywordstyle=\color{blue},%
morekeywords=[2]{1}, keywordstyle=[2]{\color{black}},
identifierstyle=\color{black},%
stringstyle=\color{mylilas},
commentstyle=\color{mygreen},%
showstringspaces=false,%without this there will be a symbol in the places where there is a space
%numbers=left,%
numberstyle={\tiny \color{black}},% size of the numbers
numbersep=9pt, % this defines how far the numbers are from the text
emph=[1]{for,end,break},emphstyle=[1]\color{red}, %some words to emphasise
%emph=[2]{word1,word2}, emphstyle=[2]{style},
frame = single
}



\title{Méthodes à haute résolution}
\author{Augustin HOFF et Gustavo CIOTTO PINTON}
\institute{ \includegraphics[scale=0.1]{images/LogoSupelec}}
\date{2014}


\begin{document}

\begin{frame}
\titlepage
\end{frame}

\begin{frame}
\frametitle{Table des matières }
\tableofcontents
\end{frame}

\section{Généralités}
\subsection{Outils mathématiques}

% ----------------------%%%%%%%%%%%

\begin{frame}

\frametitle{Outils mathématiques}
\framesubtitle{Définitions}
\begin{itemize}
  \item L'autocorrelation d'un processus aléatoire pour deux indices de temps différents \(n_1\) et \(n_2\) est définie comme
  \begin{equation}
%
\label{eq:autocorr}
r_{xx} = \mathcal E \{ x[n_1]x^*[n_2] \}
%
\end{equation}


\item La matrice formée par les valeurs d'autocorrelation:

{ \fontsize{8.5}{10}\selectfont

\begin{equation}
%
\label{eq:matricecorr}
R_{N-1}=\begin{bmatrix}
r_{xx}[0] & r_{xx}[-1] & r_{xx}[-2] & \cdots & r_{xx}[-(N-1)] \\
r_{xx}[1] & r_{xx}[0] & r_{xx}[-1] & \cdots & r_{xx}[-(N-2)] \\
\vdots & \vdots & \vdots & \ddots & \vdots \\
r_{xx}[N-1] & r_{xx}[N-2] & r_{xx}[N-3] & \cdots & r_{xx}[0] \\
\end{bmatrix}
%
\end{equation}
}

\end{itemize}
\end{frame}


% ----------------------%%%%%%%%%%%

\begin{frame}

\frametitle{Outils mathématiques}
\framesubtitle{Définitions}

\begin{itemize}
  
  \item Le rapport entre les puissances du signal et du bruit (SNR):
  
  \begin{equation}
SNR = \frac{P_{signal}}{P_{bruit}}
\end{equation}

Il est souvent écrit en \textit{décibels} [dB]:

\begin{equation}
\label{eq:snr}
SNR_{dB} = 10* \log \left[ \frac{P_{signal}}{P_{bruit}} \right]
\end{equation}

\item La puissance d'un processus aléatoire est calculée par l'équation suivante

\begin{equation}
P_x = \frac{1}{2\pi} \int_{-\pi}^{\pi}  S_x (\omega) d\omega = \int_{-\frac{1}{2}}^{\frac{1}{2}}  S_x (\nu) d\nu
\end{equation}

où \(S_x(f) = \mathcal{F} \{r_{xx}[k]\}\)

\end{itemize}
\end{frame}


% ----------------------%%%%%%%%%%%


\subsection{Application à un signal sinuisoïdal complexe}
\begin{frame}

\frametitle{Outils mathématiques}
\framesubtitle{Application à un signal sinuisoïdal complexe}

\begin{itemize}
  
  \item On pose le signal d'entrée
  
  \begin{equation}
\label{eq:x}
s[n] = \sum_{k=1}^{P} A_k \exp(j(2\pi f_k n T + \theta _k))
\end{equation}

\item

En supposant que les variables aléatoires \(\theta _k\) sont idépendantes et uniformement distribuées sur l'intervalle \( [0, 2\pi[\), on obtient alors

\begin{equation}
\label{eq:meanx}
\overline{s}[n] = 0 \text{ \hspace{5pt} et \hspace{5pt} } r_{ss}[m] = \sum_{k=1}^{P} A_k^2 \exp(j 2\pi f_k m T)
\end{equation}

De plus:

\begin{equation}
SNR_{dB} = 10* \log \left[ \frac{P_{signal}}{P_{bruit}} \right] = 10* \log \left[ \frac{\sum_{k=1}^{P}A_k^2}{P_{bruit}} \right]
\end{equation}

\end{itemize}
\end{frame}


% ----------------------%%%%%%%%%%%

\begin{frame}

\frametitle{Outils mathématiques}
\framesubtitle{Application à un signal sinuisoïdal complexe}

\begin{itemize}
  
  \item La matrice d'autocorrelation s'écrit en présence de bruit blanc de variance \(p_w\)
  
  \begin{equation}
%
\label{eq:Ryy}
R_{xx}= R_{ss} + R_w =\sum_{k=1}^{P}A_k^2 \boldsymbol{e}_{N-1}(f_k) \boldsymbol{e}_{N-1}^H(f_k) + p_w\boldsymbol{I}
%
\end{equation}

où

\begin{equation}
%
\label{eq:em}
\boldsymbol{e}_{N-1}(f_k)=\begin{bmatrix}
1 \\
\exp (2 \pi j f_k T) \\
\vdots \\
\exp (2 \pi j f_k (N-1)T) \\
\end{bmatrix}
%
\end{equation}


\end{itemize}
\end{frame}

% ----------------------%%%%%%%%%%%

\begin{frame}

\frametitle{Outils mathématiques}
\framesubtitle{Application à un signal sinuisoïdal complexe}

\begin{itemize}
  \item La matrice se réécrit à partir des P valeurs propres de \(R_{ss}\)
  
  \begin{equation}
%
\label{eq:RyyEigen}
R_{xx} = \sum_{i=1}^{P}(\lambda _i + p_w) \boldsymbol{v}_{i} \boldsymbol{v}_{i}^H + p_w\sum_{i=P+1}^{N} \boldsymbol{v}_{i} \boldsymbol{v}_{i}^H
%
\end{equation}

\item Les P premiers vecteurs propres de \(R_{xx}\) forment l'espace signal \underline {orthogonal} à l'espace bruit donné par les N-P autres vecteurs propres.

\end{itemize}
\end{frame}


% ----------------------%%%%%%%%%%%

\section{Algorithme MUSIC}
\subsection{Aspect théorique}

% ----------------------%%%%%%%%%%%

\begin{frame}

\frametitle{Algorithme MUSIC}
\framesubtitle{Aspect théorique}

\begin{itemize}
  
  \item On pose alors:
  
  { \fontsize{8.5}{10}\selectfont
  
  \begin{equation}
%
\label{eq:basemusic}
\sum_{k=P+1}^{N} \alpha _k \left| \boldsymbol{e}_{N-1}^H (f) \boldsymbol{v}_k \right| ^2 = \boldsymbol{e}_{N-1}^H (f) \left( \sum_{k=P+1}^{N} \alpha _k  \boldsymbol{v}_{k}\boldsymbol{v}_k^H \right) \boldsymbol{e}_{N-1} (f)
%
\end{equation}
}

\item Cette quantité est nulle si \(\boldsymbol{e}_{N-1}(f)\) est un des vecteurs de l'espace signal i.e si f est une fréquence du signal car \(\boldsymbol{v}_k \) est un vecteur propre de l'espace bruit pour k \( \in \llbracket P+1,N \rrbracket \)

\item L'estimateur est donc donné par l'expression

\begin{equation}
%
\label{eq:music}
P_{MUSIC} (f) = \frac{1}{\boldsymbol{e}_{N-1}^H (f) \left( \sum_{k=P+1}^{N}   \boldsymbol{v}_{k}\boldsymbol{v}_k^H \right) \boldsymbol{e}_{N-1} (f)}
%
\end{equation}

\end{itemize}
\end{frame}


% ----------------------%%%%%%%%%%%





\subsection{Mise en oeuvre}

% ----------------------%%%%%%%%%%%

\begin{frame}[fragile]
%
\frametitle{Algorithme MUSIC}
\framesubtitle{Mise en oeuvre - \textit{P\_music.m}}
%
%
\begin{itemize}
  \item  Les principales variables
  
  \begin{lstlisting}
function [ resultat ] = P_music(signal_x,  frequences , P)
N = size(signal_x ,2);
T = 1/N;
\end{lstlisting}
%
\item  Calcul de la matrice d'autocorrelation

\begin{lstlisting}
z   = corrmtx( signal_x, 0 );
Rxx = toeplitz(z);
\end{lstlisting}

\end{itemize}
\end{frame}


% ----------------------%%%%%%%%%%%

\begin{frame}[fragile]
%
\frametitle{Algorithme MUSIC}
\framesubtitle{Mise en oeuvre - \textit{test.m}}
%
%
\begin{itemize}
  
  \item Paramètres initiaux du test
  
  \begin{lstlisting}
clear all; close all; clc;
N = 100;
n = 0: 1 : N - 1;
P = 3;
T = 1/N;
fDebut = 1e0;
fFin   = 7e1;
fPas   = 1e-1;

A = [1 2.5 3];
F = [4.9e1 5e1 6e1];
Theta = 2*pi*[rand(1) rand(1) rand(1)];

SNR = [0.001 40 100 500]; %(dB)

\end{lstlisting}

\end{itemize}
\end{frame}


% ----------------------%%%%%%%%%%%

\begin{frame}[fragile]
%
\frametitle{Algorithme MUSIC}
\framesubtitle{Mise en oeuvre - \textit{test.m}}
%
%
\begin{itemize}
  
  \item Calcul des pics pour chaqu'un des SNRs
  
  \begin{lstlisting}
bAux = randn(1,N);

%signal
s = zeros (1, N);
for k = 1:P
s = s + A(k) * exp(2*pi*1i*n*T*F(k) + 1i*Theta(k));
end

%bruit
for l = 1 : size(SNR, 2)

varianceNoise = sum (A.*A) / ( 10 ^ (SNR(l) / 10));
b = sqrt(varianceNoise) * bAux;
x = s + b;

frequences = fDebut : fPas : fFin;
result = P_music (x, frequences, P);
plot(...)
end
\end{lstlisting}

\end{itemize}
\end{frame}


% ----------------------%%%%%%%%%%%

\subsection{Resultats}

\begin{frame}
%
\frametitle{Algorithme MUSIC}
\framesubtitle{Resultats}

\begin{figure}[h]
\centering
\includegraphics[scale= 0.28 ]{images/wave0001}
\caption{Signal avec du bruit pour \(SNR_{dB} = 0.001 dB\)}
\label{fig:snr0001}
\end{figure}

\end{frame}

% ----------------------%%%%%%%%%%%

\begin{frame}
%
\frametitle{Algorithme MUSIC}
\framesubtitle{Resultats}

\begin{figure}[h]
\centering
\includegraphics[scale= 0.28]{images/snr0001}
\caption{Resultats obtenus pour \(SNR_{dB} = 0.001 dB\)}
\label{fig:wave0001}
\end{figure}

\end{frame}

% ----------------------%%%%%%%%%%%

\begin{frame}
%
\frametitle{Algorithme MUSIC}
\framesubtitle{Resultats}

\begin{figure}[h]
%
\centering
\includegraphics[scale=0.28 ]{images/snr40}
\caption{Resultats obtenus pour \(SNR_{dB} = 40 dB\)}
\label{fig:snr40}
\end{figure}
\end{frame}

% ----------------------%%%%%%%%%%%

\begin{frame}
%
\frametitle{Algorithme MUSIC}
\framesubtitle{Resultats}

\begin{figure}[h]
\includegraphics[scale=0.28 ]{images/snr100}
\caption{Resultats obtenus pour \(SNR_{dB} = 100 dB\)}
\label{fig:snr100}
\end{figure}

\end{frame}

% ----------------------%%%%%%%%%%%

\begin{frame}
%
\frametitle{Algorithme MUSIC}
\framesubtitle{Resultats}

\begin{figure}[h]
\centering
\includegraphics[scale= 0.28 ]{images/wave500}
\caption{Signal avec du bruit pour \(SNR_{dB} = 500 dB\)}
\label{fig:snr0001}
\end{figure}

\end{frame}

% ----------------------%%%%%%%%%%%

\begin{frame}
%
\frametitle{Algorithme MUSIC}
\framesubtitle{Resultats}

\begin{figure}[h]
\centering
\includegraphics[scale= 0.28]{images/snr500}
\caption{Resultats obtenus pour \(SNR_{dB} = 500 dB\)}
\label{fig:wave500}
\end{figure}

\end{frame}

% ----------------------%%%%%%%%%%%

\subsection{Critère de Rayleight}
\begin{frame}
%
\frametitle{Critère de Rayleight}
\framesubtitle{Analogie avec la lumière}

\begin{figure}[h]
\centering
\includegraphics[scale= 0.22]{images/Critere_Rayleight2}
\caption{Distinction des différentes taches de lumière selon les cas}
\end{figure}

\end{frame}

% ----------------------%%%%%%%%%%%

\section{Algorithme ESPRIT}
\begin{frame}
%
\frametitle{Algorithme ESPRIT}

\begin{figure}[h]
\centering
\includegraphics[scale=.6]{images/ReseauEsprit}
\caption{Disposition des antennes.}
\end{figure}

Déphasage temporel de \( \frac{D sin(\phi) }{c}  \) où
\(\phi\) est l'angle d'incidence du signal


\end{frame}

% ----------------------%%%%%%%%%%%

\begin{frame}
%
\frametitle{Algorithme ESPRIT}

\begin{equation}
S_2 = S_1 \Omega
\end{equation}

si \[ S_1 = \begin{bmatrix}
\boldsymbol{e}_{N-1}(f_1) & \cdots & \boldsymbol{e}_{N-1}(f_P)
\end{bmatrix} \]

et \[ \Omega = \begin{bmatrix}
\exp(2\pi j f_1 \frac{D sin(\phi_1) }{c}) & & \text{\huge0}\\
&  \ddots \\
\text{\huge0} & & \exp(2\pi j f_p \frac{D sin(\phi_P) }{c})\\

\end{bmatrix} \]

\end{frame}

% ----------------------%%%%%%%%%%%

\begin{frame}
%
\frametitle{Algorithme ESPRIT}
\begin{itemize}
  
  \item La matrice d'autocorrélation s'écrit alors
  \begin{equation}
R_{xx} = \begin{bmatrix} S_1 \\ S_1 \Omega \end{bmatrix}
\begin{bmatrix} A_1^2 & & \text{0}\\ &  \ddots \\ \text{0} & & A_P^2 \\ \end{bmatrix}
\begin{bmatrix}   S_1 \\ S_1 \Omega \end{bmatrix}^H  + p_w\boldsymbol{I}
\end{equation}

\item Ce qui donne:
\begin{equation}
U = [u_1 \cdots u_P] = \begin{bmatrix} U_1 \\ U_2 \end{bmatrix} = \begin{bmatrix} S_1 \\ S_1 \Omega \end{bmatrix} T = \begin{bmatrix} S_1 T \\ S_1 \Omega T \end{bmatrix}
\end{equation}
\end{itemize}

\end{frame}

% ----------------------%%%%%%%%%%%

\begin{frame}
%
\frametitle{Algorithme ESPRIT}
\begin{itemize}
  
  \item Finalement
  \begin{equation}
U_2 = U_1 T^{-1} \Omega T = U_1 \Psi
\end{equation}

\item La méthode des moindres carrés donne alors une estimation de \( \Psi\) la matrice semblable à \( \Omega\) contenant les informations du signal

\begin{equation}
\Psi = (U_1 U_1^H)^{-1} U_1^H U_2
\end{equation}

\end{itemize}

\end{frame}

% ----------------------%%%%%%%%%%%

\section {Références}
\begin{frame}{Références}

\begin{itemize}
  \item
  Institute for Dynamics Systems and Control. \textit{White noise and power spectral density}.\\
  \url{http://www.idsc.ethz.ch/Courses/signals_and_systems/ArchiveFall10/lectureNotes8.pdf}
  
  \item S. Lawrence Marple Jr. \textit{Digital Spectral Analysis with Applications}. Prentice-Hall Inc, 1988
  
  \item Bernard Picinbono. \textit{Signaux aléatoires - Tome 2 Fonctions aléatoires et modèles avec problèmes résolus}. Dunod Université, 1993
  
  \item \textit{Les méthodes à haute résolution}. HERMES, 1998
  
  \item Yves Grenier. \textit{Méthodes haute-résolution en analyse spectrale et localisation}. \\
  \url{http://perso.telecom-paristech.fr/~rioul/liesse/2012liesse4/YG-slides.pdf}
  
\end{itemize}

\end{frame}

\end{document}