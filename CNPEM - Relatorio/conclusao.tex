\section{Conclusão}

Durante a execução do estágio, foi possível construir um servidor NTP de
\textit{stratum} 1 a partir de 3 modelos de receptores GPS. Obtemos um
sincronismo entre o pulso PPS e o \textit{clock} da \textit{BeagleBone} com
precisão na ordem de unidades de \textit{microsegundos} para todos eles. Além
disso, desenvolveu-se um sistema de sincronismo via \textit{Ethernet} a partir
de duas implementações possíveis: uma baseada em \textit{kernel modules} do
\textit{Linux} e a outra, na unidade de tempo real PRU. Apesar de não
conseguirmos \textit{jitters} da mesma ordem de grandeza em relação a outras soluções
desenvolvidas no grupo anteriormente, os resultados encontrados foram
satisfatórios. Foram encontrados \textit{jitters} de 5.72\textit{ms} e
0.96\textit{ms}, respectivamente.

\vspace{12pt}

Outro ponto importante do estágio foi a manipulação de um programa muito
utilizado no sistema de controle do UVX atualmente, o \textit{PROSAC}. Durante
as semanas que estive concentrado nele, desenvolvi dois clientes, um em Java e
outro em C, para computadores \textit{desktops} e o \textit{kit}
\textit{STM32F746G Discovery}, respectivamente. Além disso, participei na
elaboração do artigo \cite{pcapac2016}, publicado na PCaPAC2016, otimizando as
políticas e prioridades de escalonamento das \textit{threads} que o compõem. As
taxas de pulsos perdidos obtidos foram 0\%, 0.0001\% e 0.0003\% para as
frequências em operação atualmente no UVX (150Hz, 512Hz e 1kHz,
respectivamente).

\vspace{12pt}

Por fim, algumas ferramentas EPICS foram estudadas e aprimoradas. Em relação ao
\textit{EPIC Archiver Appliance}, além de sua instalação em um computador
localizado na sala de controle, foram adicionadas as funcionalidades de
\textit{login} e \textit{autenticação} de usuários, bem como a personalização
das interfaces \textit{web} de acordo com as cores do logotipo do laboratório.
Adicionalmente, fui responsável por instalar e configurar o servidor de alarmes
\textit{BEAST} e exportar uma versão do \textit{Control System Studio} com
apenas as funcionalidades desejadas no grupo, para a qual dei o nome de
\textit{LNLStudio}. Este nome produto foi baseado na última versão estável do
CSS, sendo a 4.3.4.

\vspace{12pt}

De maneira geral, entrei em contato com diversos conteúdos distintos, variando
de aplicações de baixo nível até a interfaces gráficas. Creio que as atividades
desenvolvidas no grupo de controle  contribuíram tanto para meu desenvolvimento
técnico quanto ao desenvolvimento pessoal. Julgo também que a orientação de
todos do laboratório foi fundamental para que eu pudesse aproveitar ao máximo as
atividades propostas. Tive, igualmente, a oportunidade de aprender conceitos que
não foram diretamente dados no meu curso de graduação, muitos ligados à física
de operação do acelerador.
 