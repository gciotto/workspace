\section {Sincronização via \textit{Ethernet} com a \textit{BeagleBone Black}}

\subsection {Introdução}

A sincronização dos diversos componentes presentes no sistema de controle de um
acelerador de partículas é um fator fundamental para seu bom funcionamento. Os
sistemas que necessitam sincronismo, como as fontes de corrente para os imãs,
controladores de feixe e de injeção, por exemplo, devem exercer as suas
respectivas funções em momentos especificados por pulsos de sincronismo, que
podem ser enviados, por exemplo, por geradores de sinais espalhados pela
infraestrutura local. A precisão com que estes equipamentos recebem tais pulsos
depende de diversas variáveis como, por exemplo, o tipo de canal de comunicação
utilizado no envio do pulso e a interface de entrada de dados que eles possuem.
A precisão pode ser obtida através da medição do \textit{jitter}, ou
desvio-padrão, do \textit{delay} calculado entre o envio do pulso e a sua
recepção no equipamento.
O \textit{Sirius}, por exemplo, possui especificações para diversos sistemas de
sincronismo, que variam de acordo com a necessidade de precisão dos equipamentos
e das funções desempenhadas por eles. Para casos mais graves, como o da bomba de
elétrons do acelerador linear (\textit{LINAC}), os \textit{jitters} não podem
ultrapassar os \textit{50ps} e soluções especiais devem ser exploradas, como a
implementação de uma rede \textit{WhiteRabbit}, que está sendo
desenvolvida e estudada por outros laboratórios no mundo todo.

\vspace{12pt}

Um sistema de sincronismo para as fontes de corrente já
foi implementado pelo grupo de controle, utilizando as unidades
\textit{Programmable Realtime Unit}, ou simplesmente \textit{PRU}, presentes na
\textit{BeagleBone Board}. Tais unidades apresentam \textit{clock} de 200MHz,
núcleos de memória própria e compartilhada, e módulos dedicados, como o
\textit{Enhanced GPIO}, que favorecem a implementação de aplicações
\textit{realtime}, em que a precisão é uma necessidade importante. Soluções
implementadas para a \textit{PRU} não estão sujeitas a fatores que degradam a
precisão como o compartilhamento de \textit{CPU} com outros
processos e preeempção. Aplicações que rodam sobre \textit{Linux} embarcado e
que, portanto, compartilham recursos com outros processos, apresentam,
geralmente, desempenho inferior. O sistema desenvolvido utiliza a \textit{PRU}
para a recepção do sinal de sincronismo e ativa, posteriormente, seus nós
escravos, conforme figura \ref{fig:pru_sincronismo} abaixo.

\vspace{12pt} 

Nesta seção, serão apresentadas algumas alternativas a esse sistema, que
utilizam o protocolo \textit{Ethernet} para o envio dos \textit{triggers} de
sincronismo, e as suas respectivas \textit{performances}.

\subsection {Implementação}

