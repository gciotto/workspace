\section{Servidores de variáveis EPICS}
\label{sec:pcaspy}

\subsection{Introdução}

\textit{EPICS}, do inglês \textit{Experimental Physics and Industrial Control
System}, é um conjunto de ferramentas e aplicações que permitem monitorar e
controlar sistemas que possuem, além de milhares de variáveis, uma larga e
complexa rede ligando centenas de computadores e equipamentos. Tal sistema já é
utilizado em diversos outros síncrotons e laboratórios no mundo todo e o seu
desenvolvimento é realizado cooperativamente entre tais organizações. O
principal intuito é proporcionar aos operadores de sistemas de grande porte uma
visão global dos diversos módulos, facilitando, desta maneira, a identificação e
previsão de problemas, assim como a escolha da melhor ação corretora. Entre as
ferramentas oferecidas, destaca-se, por exemplo, arquivadores de variáveis,
monitores de alarmes e suporte para construção de \textit{interfaces} gráficas
específicas para determinadas aplicações, que serão discutidos nas próximas
seções deste documento.

\vspace{12pt}

Atualmente, no \textit{UVX}, o controle é realizado através de módulos
independentes, muitas vezes implementados em linguagens de programação
distintas, que fornecem soluções específicas para um determinado equipamento e
baixo grau de interação com outros programas. A identificação de problemas é,
portanto, dificultada e depende muito mais de experiências anteriores dos
operadores. Sendo assim, a implementação de \textit{EPICS} para o sistema
controle do \textit{Sirius} está sendo considerada fortemente. Além disso, o
acelerador linear de eléctrons, que está sendo atualmente desenvolvido por uma
empresa chinesa, a ser utilizado neste novo síncroton também oferecerá  uma
\textit{interface} de comunicação com total compatibilidade com aplicações
\textit{EPICS}.

\vspace{12pt}

De maneira simples, um sistema \textit{EPICS} é composto por diversos
servidores e clientes espalhados pela rede, capazes, respectivamente, de
publicar ou obter os estados dos diversos dispositivos pertencentes à estrutura.
Tais estados são chamados de variáveis, ou somente \texttt{PVs}, e podem
assumir vários tipos (binário, inteiro, \textit{float} \ldots), faixas de
operação e taxas de aquisição que dependem principalmente da maneira de como os
equipamentos proveêm tais dados. Cabe aos servidores (chamados também de
\textit{Input/Output Controllers} ou \textit{IOCs}), portanto, realizar as
operações de escrita ou leitura no \textit{mundo real} ou obter tais informações
de outros dispositivos, como osciloscópios e sensores, e publicá-las aos
clientes. A comuncicação entre clientes e servidores é realizado através do
protocolo de rede \textit{Channel Access}, implementado para redes que oferecem
grande largura de bandura, capaz de garantir o atendimento de especificações de
\textit{soft real-time} (consultar seção \ref{sec:bb1} para mais informações).

\vspace{12pt}

As próximas subseções serão dedicadas ao detalhamento de duas possíveis
implementação de servidores de variáveis. A primeira utiliza a biblioteca
\path{pcaspy} de \textit{Python}, enquanto que a outra, um módulo chamado
\path{StreamDevice}.

\subsection{Utilizando a biblioteca \texttt{PCASpy}}

O módulo \path{PCASpy} (\path{PCAS} do inglês \textit{Python Channel Access
Server}) fornece classes que realizam tanto a comunicação com um cliente
\textit{Channel Access} quanto com os dispositivos medidores. Esse módulo
oferece quatro classes, porém, em princípio, utiliza-se apenas duas, sendo elas
\path{SimpleServer} e \path{Driver}.

\vspace{12pt}

A classe \path{SimpleServer} abstrai os detalhes da comunicação entre servidor e
clientes e redireciona requisições de leitura e escrita para um objeto da classe
\path{Driver}, que por sua vez, se comunica com o dispositivo medidor. Um objeto
da classe \path{SimpleServer}, portanto, é a ponte entre os clientes e tais
equipamentos. Entre seus métodos, destaca-se:

\begin{itemize}
  \renewcommand\labelitemi{--}
  \item \texttt{createPV(\textit{prefix}, \textit{pvdb})}: registra e torna
  disponível as variáveis contidas no dicionário \texttt{\textit{pvdb}}. A
  documentação do módulo na \textit{Internet} contem os campos válidos para
  configuração de uma \texttt{\textit{PV}}.

  \item \texttt{process(\textit{time})}: o servidor processa requisições dos
  clientes a cada \texttt{\textit{time}} segundos.
\end{itemize}

A classe \texttt{Driver} deve ser redefinida a fim de refletir os detalhes de
comunicação e protocolo.  Para tal, através de mecanismos de herança,
redefine-se os métodos \texttt{\textit{read()}} e \texttt{\textit{write()}}, que
realizam, respectivamente, a leitura e escrita de variáveis. Além destas duas
funções, essa classe mantém um registro interno dos valores das variáveis. Para
modificá-lo, dois outros métodos são usados, sendo eles
\texttt{\textit{getParam()}} e \texttt{\textit{setParam()}}. Enfim, a função
\texttt{\textit{updatePV()}} é chamada para comunicar ao objeto
\path{SimpleServer} que uma \textit{PV} foi modificada.

\subsection{\textit{EPICS StreamDevice}}

O EPICS \textit{StreamDevice} oferece suporte para a comunicação com
dispositivos que podem ser controlados através do envio de \textit{strings},
isto é, cadeias de \textit{bytes}. Dessa forma, dispositivos que possuem
interface \textit{serial} de dados, como os padrões RS-232 e RS-485, podem se
comunicar com um \textit{StreamDevice} e fornecer os valores para as diversas
variáveis EPICS configuradas.

\vspace{12pt}

A configuração de um \textit{StreamDevice} é realizada a partir de arquivos
ditos do tipo \textit{protocolo}. Tais arquivos especificam os formatos e os
padrões que são entendidos pelo dispositivo e determinam, portanto, os
\textit{bytes} que devem ser enviados para cada tipo de operação. Muitas opções
de comandos e formatos são disponíveis nesta ferramenta, sendo que as principais
serão detalhadas nas próximas subseções. As variáveis que são disponibilizadas à
rede são, por sua vez, especificadas no arquivo \textit{database}, que é lido
durante a inicialização de um \textit{StreamDevice}. As operações de leitura e
escrita destas respectivas variáveis devem estar presentes em um dos arquivos de
\textit{protocolo}.

\vspace{12pt}

A comunicação das mensagens para o meio físico propriamente dita não é realizada
pelo \textit{StreamDevice}. Por padrão, tal ferramenta possui uma
\textit{interface} com um \textit{driver} de comunicação assíncrona chamado
\textit{asynDriver}, que deve ser instalado separadamente. De modo prático, o
módulo \textit{StreamDevice} é responsável por converter os \textit{bytes}
recebidos e enviados em variáveis EPICS, enquanto que o \textit{asynDriver}
realiza a comunicação física com os dispositivos. É importante notar que é
o uso do \textit{asynDriver} não é imposto pelo \textit{StreamDriver}: é
possível, portanto, substituí-lo por qualquer outro \textit{driver} de
comunicação caso desejado.

\vspace{12pt}

As próximas subseções são dedicadas ao processo de instalação e configuração de
um \textit{StreamDevice} para a bomba de vácuo \textit{Agilent 4UHV}, que será
utilizada no projeto \textit{Sirius}.

\subsubsection{Instalação do \textit{StreamDevice}}

A primeira etapa consiste na instalação do \textit{asynDriver}. Para isso, é
necessário seguir os seguintes itens abaixo. Supõe-se que a biblioteca de
funções EPICS já foi instalada conforme será apresentado na seção
\ref{ref:epics-install}. O diretório onde os arquivos do \textit{asynDriver}
serão instalados será referenciado por \path{$INSTALL_DIR/asyn4-29/}.

\begin{itemize}
  \renewcommand\labelitemi{--}
  \item Faça o \textit{download} do pacote através de 
  \url{http://tinyurl.com/hy3sra7} e descompacte-o.
  \item Execute o arquivo \path{Makefile} através do comando \path{make}. Os
  diversos que compõem o \textit{driver} serão compilados nesta fase.
\end{itemize}

O \textit{StreamDevice} está disponível a partir do \textit{link}
\url{http://tinyurl.com/z8rkc7y}.

