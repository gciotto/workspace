\section {Manutenção e Implementação de Clientes PROSAC}

\subsection{Introdução}

O \textit{PROSAC} é um \textit{firware} desenvolvido no grupo de controle, cujo
principal propósito é receber requisões da sala de controle e acionar a
respectiva placa do bastidor. Esse \textit{firware} apresenta, atualmente,
suporte a diversas placas que são usadas atualmente no \textit{UVX} para
controle de fontes e monitoramento de sensores. Exemplos de placas operadas pelo
\textit{PROSAC} são a \textit{LOCON} de 12 e 16 \textit{bits}, a
\textit{STATFNT} e a \textit{DIGINT}.

\vspace{12px}

Considerando a sua importância no contexto do sistema controle, dois clientes
foram implementados a fim de testar seu funcionamento: um em Java e outro,
utilizando o \textit{kit} de desenvolvimento \textit{STM32F7 Discovery}.

\subsection {Manutenção do cliente \textit{PROSAC} escrito em \textit{Java}}

O cliente \textit{PROSAC} implementado em \textit{Java} foi desenvolvido em 2011
por Bruno Martins para testes iniciais do \textit{PROSAC}. Desta forma, casos mais
complexos, como placas desenvolvidas posteriormente ao cliente, produziam
exceções que interrompiam o programa. As seguintes modificações foram realizadas
a fim de corrigir tais problemas:

\begin {enumerate} [i.] 
  \item Função \texttt{processCommand} da classe \texttt{Client}: foi adicionado
  uma condição para verificar se a quantidade de \textit{bytes} recebidos do
  \textit{PROSAC} é a mesma que a esperada. Tal condição encontra-se no laço
  \texttt{for} do bloco \texttt{default} do \texttt{switch}.

  \item Modificação de forma que os \textit{status} das placas apareçam em uma
  janela distinta daquela onde o painel de comando está inserido. A classe
  \texttt{Boards} foi substituída pela \texttt{BoardsFrame}.
  
  \item Suporte às placas \textit{Statfnt}, \textit{Digint}, \textit{Rux} 12
  \textit{bits} bipolar e \textit{Mux} 16 \textit{bits}, e implementação das
  respectivas interfaces gráficas.

  \item Correção nas escalas dos gráficos para as placas monopolares de 12 e 16
  \textit{bits}.
  
  \item Para a operação de rampa, o \textit{PROSAC} necessita da ordem com que
  as placas aparecem no bastidor, não seus \textit{IDs}. Por exemplo, se duas
  placas estiverem presentes, cujos \textit{IDs} são 5 e 19, e quisermos
  executar uma rampa na 5, temos que enviar sua posição do bastidor, isto é 0.

\end{enumerate}

\subsection {Implementação para o \textit{kit} \textit{STM32F7 Discovery}}

O \textit{kit} \textit{STM32F7 Discovery} oferece diversos recursos
como interface \textit{Ethernet}, \textit{I2C}, \textit{UART} e tela
\textit{LCD Touch} capacitiva. Os principais passos na implementação desta
solução foram:

\begin{enumerate} [i.]
  \item Configuração de plugins para desenvolvimento na \textit{IDE Eclipse}.
  \item Configuração da interface \textit{OpenOCD}, responsável pela comunicação
  entre placa e computador.
  \item Implementação de um projeto na aplicação \textit{STMCubeMX} para
  inicialização dos pinos dos módulos que serão utilizados no projeto.
  \item Adição dos \textit{middlewares} \textit{FreeRTOS} e \textit{LwIP} ao
  projeto.
  \item Correção do \path{stm32f7_hal_conf.h} com a definição dos
  registradores corretos do módulo \textit{PHY}.
  \item Criação de uma interface gráfica, capaz de reconhecer
  eventos de \textit{touch}.
\end{enumerate}

A lógica utilizada nesta aplicação foi adaptada do cliente \textit{Java} e,
portanto, apresenta os mesmos comportamentos. Duas \textit{threads} são criadas,
sendo que uma envia requisições de leitura de dados a todo momento ao
\textit{PROSAC} e a outra, comandos requisitados pelo usuário, como ciclagem e
rampa.
