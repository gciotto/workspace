\section {Manutenção do PROSAC e Implementação de Aplicações Clientes}

\subsection{Introdução}

O \textit{PROSAC} é um \textit{firmware} desenvolvido no grupo de controle, cujo
principal propósito é receber requisições da sala de controle e acionar a
respectiva placa do bastidor. Esse \textit{firmware} apresenta, atualmente,
suporte a diversas placas que são usadas atualmente no \textit{UVX} para
controle de fontes e monitoramento de sensores. Exemplos de placas operadas pelo
\textit{PROSAC} são a \textit{LOCON} de 12 e 16 \textit{bits}, a
\textit{STATFNT} e a \textit{DIGINT}.

\vspace{12px}

Considerando a sua importância no contexto do sistema controle, um artigo
foi publicado na PCaPAC 2016 \cite{pcapac2016} e dois clientes foram
implementados a fim de testar seu funcionamento: um em Java e outro, utilizando o \textit{kit} de
desenvolvimento \textit{STM32F7 Discovery}.

\subsection {Manutenção do \textit{PROSAC}}

O grupo de controle publicou na PCaPAC 2016 um artigo entitulado \textit{UVX
Control System: An Approach With BeagleBone Black} \cite{pcapac2016}, em que fui
um dos co-autores. Minha tarefa neste trabalho, em poucas palavras, foi ajustar
a política de escalonamento e a prioridade das \textit{threads} que compõem o
\textit{PROSAC}, de forma a obter o melhor desempenho possível nas tarefas
síncronas realizadas pelo programa. As tarefas ditas \textit{síncronas}
são aquelas que necessitam ser sincronizadas a partir de um pulso de sincronismo
externo. No \textit{PROSAC}, tais tarefas consistem nas operações de ciclagem e
rampa, isto é, a cada novo pulso recebido, o \textit{PROSAC} é encarregado de
transmitir um novo valor às placas que estiverem participando do processo.
Devido às mesmas limitações de processamento de tempo real discutidas na seção
\ref{sec:bb1}, o número de pulsos perdidos na nova versão do \textit{PROSAC} era
elevado para as frequências superiores a 500Hz, que tornava o seu uso
proibitivo. Dessa forma, propus utilizarmos o \textit{scheduler} denominado de
\textit{Completely Fair Scheduler} e ajustar a prioridade das \textit{threads}
de forma a garantir um maior tempo de CPU à \textit{thread} responsável por
processar os pulsos. A tabela \ref{tab:prosac} a seguir ilustra a taxa de pulsos
perdidos para a configuração do \textit{PROSAC} com o \textit{Completely Fair
Scheduler} conectado a duas placas LOCON de \textit{12 bits}, ambas
participando da ciclagem. Conforme tabela, as taxas de pulsos perdidos são muito
inferiores aos pulsos totais recebidos, possibilitando, assim, seu uso no UVX.
A documentação interna do Grupo contém resultados de testes realizados para a
versão anterior do PROSAC, que roda sobre uma CPU \textit{Advantech PCM-4153F}, para
duas frequências, 500Hz e 5kHz. Nesta oportunidade, foram obtidas
taxas de pulsos perdidos de 0\% e 0.20\% \cite{martins}, respectivamente. A taxa
encontrada para 500Hz é inferior àquela encontrada no nosso artigo, porém a
quantidade total de pulsos enviados foi 82 vezes menor: os testes utilizaram
apenas 65536 pulsos. Destaca-se também que o uso da \textit{Beagle} não
justifica-se apenas pela quantidade de pulsos perdidos, mas também pelo seu
custo (a \textit{Beagle} custa \$50, enquanto que a \textit{Advantech}, \$500) e pela sua
forte comunidade de desenvolvedores.

\begin{table}[h]

	\centering
	\caption{\label{tab:prosac} Resultados do \textit{PROSAC} com o
	\textit{Completely Fair Scheduler}.}
	\begin{tabular}{| c | c | c | c |}
		\hline
		\textbf{Frequência (Hz)} & \textbf{Pulsos perdidos} & \textbf{Total} &
		\textbf{Razão} \\ \hline 
		150 & 0 & 579.812  & 0 \\ \hline
		512 & 8 & 5.376.056 & 0.0001\% \\ \hline
		1000 & 18 & 6.050.225 & 0.0003\% \\ \hline
	\end{tabular}	    
\end{table}
 
 
\subsection {Manutenção do cliente \textit{PROSAC} escrito em \textit{Java}}

O cliente \textit{PROSAC} implementado em \textit{Java} foi desenvolvido em 2011
por Bruno Martins para testes iniciais do \textit{PROSAC}. Desta forma, casos mais
complexos, como placas desenvolvidas posteriormente ao cliente, produziam
exceções que interrompiam o programa. As seguintes modificações foram realizadas
a fim de corrigir tais problemas:

\begin {enumerate} [i.] 
  \item Função \texttt{processCommand} da classe \texttt{Client}: foi adicionada
  uma condição para verificar se a quantidade de \textit{bytes} recebidos do
  \textit{PROSAC} é a mesma que a esperada. Tal condição encontra-se no laço
  \texttt{for} do bloco \texttt{default} do \texttt{switch}.

  \item Modificação de forma que os \textit{status} das placas apareçam em uma
  janela distinta daquela onde o painel de comando está inserido. A classe
  \texttt{Boards} foi substituída pela \texttt{BoardsFrame}.
  
  \item Suporte às placas \textit{Statfnt}, \textit{Digint}, \textit{Rux} 12
  \textit{bits} bipolar e \textit{Mux} 16 \textit{bits}, e implementação das
  respectivas interfaces gráficas.

  \item Correção nas escalas dos gráficos para as placas monopolares de 12 e 16
  \textit{bits}.
  
  \item Para a operação de rampa, o \textit{PROSAC} necessita da ordem em que
  as placas aparecem no bastidor, não seus \textit{IDs}. Por exemplo, se duas
  placas estiverem presentes, cujos \textit{IDs} são 5 e 19, e quisermos
  executar uma rampa na 5, temos que enviar sua posição do bastidor, isto é 0.

\end{enumerate}

\subsection {Implementação para o \textit{kit} \textit{STM32F7 Discovery}}

O \textit{kit} \textit{STM32F7 Discovery} oferece diversos recursos
como interface \textit{Ethernet}, \textit{I2C}, \textit{UART} e tela
\textit{LCD Touch} capacitiva. Os principais passos na implementação desta
solução foram:

\begin{enumerate} [i.]
  \item Configuração de \textit{plugins} para desenvolvimento na \textit{IDE
  Eclipse}.
  \item Configuração da interface \textit{OpenOCD}, responsável pela comunicação
  entre placa e computador.
  \item Implementação de um projeto na aplicação \textit{STMCubeMX} para
  inicialização dos pinos dos módulos que serão utilizados no projeto.
  \item Adição dos \textit{middlewares} \textit{FreeRTOS} e \textit{LwIP} ao
  projeto.
  \item Correção do \path{stm32f7_hal_conf.h} com a definição dos
  registradores corretos do módulo \textit{PHY}.
  \item Criação de uma interface gráfica, capaz de reconhecer
  eventos de \textit{touch}.
\end{enumerate}

A lógica utilizada nesta aplicação foi adaptada do cliente \textit{Java} e,
portanto, apresenta os mesmos comportamentos. Duas \textit{threads} são criadas,
sendo que uma envia requisições de leitura de dados a todo momento ao
\textit{PROSAC} e a outra, comandos requisitados pelo usuário, como ciclagem e
rampa.
