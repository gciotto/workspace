\section {Introdução}

Durante o primeiro semestre de 2016, desenvolvi minhas atividades de estágio no
Laboratório Nacional de Luz Sincroton (LNLS - CNPEM) no grupo de Controle, que
é responsável por prover soluções de controle e sensoriamento de diversos
equipamentos utilizados no acelerador de partículas, como fontes de tensão e
corrente, bombas de vácuo e sensores de temperatura, por exemplo. É
responsabilidade do grupo, igualmente, planejar e desenvolver as futuras
ferramentas que serão utilizadas no acelerador \textit{Sirius}, que se encontra em fase de construção
atualmente.

\vspace{12pt}

Nestes 5 meses, entrei em contato com o desenvolvimento
de \textit{software} e \textit{hardware} para sistemas embarcados, especialmente
para Linux embarcado, já que o laboratório pretende utilizar a
\textit{BeagleBone Black} para futuras implementações. Entre os principais
projetos, pude desenvolver módulos para o \textit{kernel} do Linux, programas
para o módulo \textit{realtime} da Beagle e integrar um receptor GPS a esta
mesma placa. As próximas seções são dedicadas aos detalhes de implementação
destes e de outros problemas.

\vspace{12pt}

Cabe destacar que este relatório trata-se de uma versão ainda não completa,
visto que algumas atividades deverão ser finalizadas e, portanto, documentadas
somente no segundo semestre de 2016.

\newpage
