\section {Introdução}

Desde fevereiro de 2016, desenvolvi minhas atividades de
estágio no Laboratório Nacional de Luz Síncrotron (LNLS - CNPEM) no grupo de
Controle, que é responsável por prover soluções de controle e supervisão de diversos
equipamentos utilizados no acelerador de partículas, como fontes de tensão e
corrente, bombas de vácuo e sensores de temperatura, por exemplo. É
responsabilidade do grupo, igualmente, planejar e desenvolver as futuras
ferramentas de controle que serão utilizadas no acelerador \textit{Sirius}, que
se encontra em fase de construção.

\vspace{12pt}

Nestes 10 meses, entrei em contato com desenvolvimento
de \textit{software} e \textit{hardware} para sistemas embarcados, especialmente
para Linux embarcado, já que o Grupo pretende utilizar a
\textit{BeagleBone Black} para suas futuras
implementações. Entre os principais projetos, pude desenvolver módulos para o
\textit{kernel} do Linux, programas para os núcleos \textit{realtime} da Beagle
e integrar um receptor GPS a esta mesma placa a fim de disponibilizar um servidor NTP de \textit{stratum} 1 à
rede de controle. Além disso, entrei em contato com algumas ferramentas
relacionadas ao sistema de controle EPICS, tais como o
\textit{Control System Studio}, o arquivador \textit{EPICS Archiver Appliance}
e o servidor de alarmes \textit{BEAST}. As próximas seções são dedicadas aos
detalhes de implementação destes e de outros projetos.

\newpage
