\section {EPICS Archiver Appliance}
\label{sec:archiver}
\subsection {Introdução}

O EPICS \textit{Archiver Appliance}, desenvolvido pelo instituto americano
\textit{National Accelerator Laboratory (SLAC)}, é capaz de monitorar e arquivar
um grande número de váriaveis, as chamadas \textit{PVs}, geradas por servidores
EPICS presentes na rede. O sistema fornece também opções de configuração de um
largo conjunto de parâmetros referentes ao armazenamento e monitoramento. Uma
\textit{appliance} é composta basicamente por quatro módulos distintos, sendo eles:

\begin{itemize}
  \renewcommand\labelitemi{--}
  \item \textit{Management}: provê as ferramentas necessárias para a gerência
  da \textit{appliance}. Permite, por exemplo, adicionar ou remover \textit{PVs}
  à lista de variáveis a serem arquivadas;
  \item \textit{Engine}: realiza a integração entre os módulos;
  \item \textit{Data Retrieval}: módulo responsável por recuperar os dados das
  \textit{PVs} arquivadas;
  \item \textit{ETL}: responsável por extrair os dados e tranformá-los a fim de
  que as aplicações possam processá-los posteriormente;
\end{itemize}

A figura \ref{fig:epics_archiver} esquematiza o modo de funcionamento do
\textit{EPICS Archiver Appliance}.

\begin{figure}[h]
    
    \centering
    \includegraphics[scale=0.6]{image/applarch}
    \caption {Modo de funcionamento de uma \textit{appliance}}
    \label{fig:epics_archiver} 
\end{figure} 

\vspace{12pt}

O instituto desenvolvedor da aplicação sugere que cada módulo seja lançada em
sua própria instância \textit{Tomcat}. Em adição, ele propõe a divisão da
unidade de armazenamento em 3 outras unidades, de acordo com a frequência que os
dados são salvos. Essas unidades são divididas \textit{short-term},
\textit{medium-term} e \textit{long-term storage}, cujas frequências de
armazenamento são, respectivamente, a cada hora, diária e anual. Essas
configurações podem ser modificadas através da modificação de arquivos
específicos, explicados nas próximas subseções.

\vspace{12pt}


Em um ambiente composto por diversos servidores EPICS e milhares de variáveis a
serem monitoradas, tal como o \textit{Sirius}, um sistema capaz de automizar e
agilizar o armazenamento e recuperação de dados se torna fundamental para o
monitoramento de eventuais problemas. Sendo assim, as próximas seções são
dedicadas à instalação e exploração dos recursos disponíveis nesta aplicação.
\subsection {Instalação}

Esta subseção é dedicada às etapas necessárias para a instalação do EPICS
Archiver Appliance. Neste tutorial serão apresentadas dois métodos: um que
utiliza os repositórios oficiais do Ubuntu 16.04 para obter os pacotes e o
outro, os servidores dos próprios provedores dos serviços.

\subsubsection {Instalação das dependências}
\label{ref:epics-install}
A aplicação necessita, além da própria base de bibliotecas EPICS, da
instalação do \textit{java-jdk 8} e de servidores \textit{MySQL} e
\textit{Apache Tomcat}. A instalação do \textit{Apache HTTP Server} é opcional e
deve ser utilizada somente se o balanceamento de requisições entre as
\textit{appliances} for pretentida. Nesse caso, vamos utilizar o módulo
\textit{mod\_proxy\_balancer}.

\begin {enumerate}[i.] 
  \item \textit{Instalação do EPICS}: é necessário inicialmente fazer o
  \textit{download} e compilar as bibliotecas EPICS. Escolha um diretório de
  acordo com sua preferência (referenciado nesse documento por
  \texttt{\$EPICS\_DIR}) e execute os comandos abaixo. Alguns erros podem
  ocorrer na execução do comando \texttt{make}, resultantes da falta de algumas
  bibliotecas no sistema. Faça as respectivas instalações e repita o processo.
\begin{lstlisting}[keywordstyle=\ttfamily, style=nonumbers]
$ cd $EPICS_DIR
$ wget http://www.aps.anl.gov/epics/download/base/baseR3.14.12.5.tar.gz
$ tar -xvzf baseR3.14.12.5.tar.gz
$ rm baseR3.14.12.5.tar.gz
$ cd base-3.14.12.5
$ make
\end{lstlisting}

Adicione as seguintes variáveis de ambiente ao arquivo \texttt{.bashrc} do seu
usuário (geralmente encontrado em \texttt{/home/user/}). Esse arquivo de
configuração é consultado toda vez que um usuário executa um novo
\textit{shell}.
\begin{lstlisting}[keywordstyle=\ttfamily, style=nonumbers]
export PATH=$EPICS_DIR/base-3.14.12.5/bin/linux-x86_64:$PATH
export EPICS_BASE=$EPICS_DIR/base-3.14.12.5
export EPICS_HOST_ARCH=linux-x86_64
\end{lstlisting}	

Enfim, atualize a sessão com

\begin{lstlisting}[keywordstyle=\ttfamily, style=nonumbers]
$ source /home/user/.bashrc
\end{lstlisting}	

Para testar a instalação, use o comando \texttt{caget} com
alguma variável disponível na sua rede.

\begin{lstlisting}[keywordstyle=\ttfamily, style=nonumbers]
$ caget variavel
\end{lstlisting}

\item \textit{Instalação do Java 8}: execute o comando abaixo para instalar o
\textit{Java Development Kit}.
\begin{lstlisting}[keywordstyle=\ttfamily, style=nonumbers]
$ sudo apt-get install openjdk-8-jdk
\end{lstlisting}

O EPICS Archiver Appliance necessita de duas variáveis de ambiente referentes ao
\textit{Java}. Adicione-as no arquivo \textit{.bashrc}, após as variáveis de
ambiente EPICS.

\begin{lstlisting}[keywordstyle=\ttfamily, style=nonumbers]
export JAVA_HOME=/usr/lib/jvm/java-8-openjdk-amd64
export JRE_HOME=${JAVA_HOME}/jre
\end{lstlisting}

\item \textit{Instalação do MySQL Server}: para instalar um servidor
\textit{MySQL}, basta executar o comando abaixo. Durante o processo de
instalação, será requisitada a senha para o usuário \textit{root} do servidor.
No nosso caso, senha escolhida foi \texttt{toor}.
\begin{lstlisting}[keywordstyle=\ttfamily, style=nonumbers]
$ sudo apt-get install mysql-server
\end{lstlisting}


\item \textit{Instalação do Apache Tomcat}: defina um diretório onde os arquivos
serão instalados e execute os comandos abaixo. Esse diretório será referenciado
pela variável de ambiente \path{${TOMCAT_DIR}}.

\begin{lstlisting}[keywordstyle=\ttfamily, style=nonumbers]
$ cd ${TOMCAT_DIR}
$ wget http://tinyurl.com/zewzg72
$ tar -xvzf apache-tomcat-8.5.4.tar.gz
$ rm apache-tomcat-8.5.4.tar.gz
$ cd apache-tomcat-8.5.4/
\end{lstlisting}

Defina a variável de ambiente \texttt{\$\{CATALINA\_HOME\}}, responsável
por referenciar o diretório \textit{apache-tomcat-8.5.4/} recém criado. Adicione
a seguinte linha no arquivo \texttt{.bashrc} logo abaixo das variáveis de ambiente
adicionadas na instalação do \textit{Java}.

\begin{lstlisting}[keywordstyle=\ttfamily, style=nonumbers]
export CATALINA_HOME = ${TOMCAT_DIR}/apache-tomcat-8.5.4
\end{lstlisting}

No \textit{Ubuntu} 16.04, a instalação pode ser realizada com o comando 

\begin{lstlisting}[keywordstyle=\ttfamily, style=nonumbers]
sudo apt-get install tomcat8
\end{lstlisting}

E defina a variável \texttt{\$\{CATALINA\_HOME\}} como 

\begin{lstlisting}[keywordstyle=\ttfamily, style=nonumbers]
export CATALINA_HOME = /usr/share/tomcat8
\end{lstlisting}

\item \textit{Instalação do Apache HTTP Server}: Para instalar o servidor
\textit{Apache}, faça o \textit{download} do pacote. 

\begin{lstlisting}[keywordstyle=\ttfamily, style=nonumbers]
$ cd $INSTALL_DIR
$ wget http://tinyurl.com/z8gnxmk
$ tar -xvzf httpd-2.4.23.tar.gz
$ rm httpd-2.4.23.tar.gz
$ cd httpd-2.4.23
\end{lstlisting}

Algumas dependências devem ser instaladas:

\begin{lstlisting}[keywordstyle=\ttfamily, style=nonumbers]
$ sudo apt-get install libapr1-dev libaprutil1-dev
\end{lstlisting}

Será necessário instalar também a \texttt{libpcre}:

\begin{lstlisting}[keywordstyle=\ttfamily, style=nonumbers]
$ cd $INSTALL_DIR
$ wget ftp://ftp.csx.cam.ac.uk/pub/software/programming/pcre/pcre-8.37.tar.gz
$ tar -xvzf pcre-8.37.tar.gz
$ rm pcre-8.37.tar.gz
$ cd pcre-8.37
$ ./configure --prefix=$INSTALL_DIR/pcre
$ make
$ make install
\end{lstlisting}

Finalmente, execute os \textit{scripts} para instalar o \textit{Apache http}:

\begin{lstlisting}[keywordstyle=\ttfamily, style=nonumbers]
$ cd $INSTALL_DIR/httpd-2.4.23
$ ./configure --prefix=$INSTALL_DIR/httpd --with-pcre=$INSTALL_DIR/pcre
$ make
$ make install
\end{lstlisting}

Para habilitar o módulo \texttt{mod\_proxy\_balancer}, descomente as seguintes
linhas no arquivo \texttt{\$INSTALL\_DIR/httpd/conf/httpd.conf}:

\begin{lstlisting}[keywordstyle=\ttfamily, style=nonumbers]
LoadModule proxy_module modules/mod_proxy.so
LoadModule proxy_http_module modules/mod_proxy_http.so
LoadModule proxy_balancer_module modules/mod_proxy_balancer.so
LoadModule lbmethod_byrequests_module modules/mod_lbmethod_byrequests.so
\end{lstlisting}

Adicione as linhas abaixo no mesmo arquivo. Quando uma requisição for realizada
pelo endereço \texttt{localhost/lnls-archiver}, um dos endereços dentro da
\textit{tag} \texttt{\(<\)Proxy\(>\)} será chamado. No nosso exemplo, colocamos
o endereço do módulo \texttt{retrieval} de apenas uma \textit{appliance}.

\begin{lstlisting}[keywordstyle=\ttfamily, style=nonumbers]
<Proxy "balancer://appl">
    BalancerMember "http://localhost:11998/retrieval"
</Proxy>
ProxyPass "/lnls-archiver" "balancer://appl"
ProxyPassReverse "/test" "balancer://appl"
\end{lstlisting}

Inicie o servidor com o comando:

\begin{lstlisting}[keywordstyle=\ttfamily, style=nonumbers]
$ sudo $INSTALL_DIR/httpd/bin/apachectl start
\end{lstlisting}

O servidor \textit{Apache} funciona agora como um divisor de carga entre as
diversas \textit{appliances} da rede.

\vspace{12pt}

No \textit{Ubuntu} 16.04, a instalação é realizada através do comando abaixo. A
configuração deste módulo é um pouco distinta daquela documentada anteriormente,
sendo que é necessário copiar os arquivos \path{proxy-balancer.*},
\path{proxy.*}, \path{proxy-http.*}, \path{status.*} e
\path{lbmethod_byrequest.*} da pasta \path{mods-avaiable/} para
\path{mods-enabled/}. Os dois últimos diretórios encontram-se em
\path{/etc/apache2/}. Enfim, a configuração dos \textit{proxies} deve ser
realizada no arquivo \path{proxy.conf} da mesma maneira que foi explicada
anteriormente.

\begin{lstlisting}[keywordstyle=\ttfamily, style=nonumbers]
$ sudo apt-get install apache2
\end{lstlisting}

O servidor é iniciado, por sua vez, através de 

\begin{lstlisting}[keywordstyle=\ttfamily, style=nonumbers]
$ sudo systemctl start apache2.service
\end{lstlisting}

\end{enumerate} 

\subsubsection {Instalação do EPICS Archiver Appliance}
\label{sec:epicsappl}

Tendo instalado todas as dependências, é necessário ainda configurar
alguns parâmetros e instalar os pacotes referentes ao arquivador. Para tal, crie
uma pasta onde serão instalados os pacotes. Esse diretório será referenciado por
\texttt{\$INSTALL\_DIR} neste documento.

\begin {enumerate}[i.] 
  \item Crie uma pasta e faça o \textit{download} dos pacotes do arquivador,
  segundo os comandos abaixo. 
\begin{lstlisting}[keywordstyle=\ttfamily, style=nonumbers]
$ mkdir $INSTALL_DIR/archappl_v0.0.1_SNAPSHOT_22-June-2016T10-25-27
$ cd $INSTALL_DIR/archappl_v0.0.1_SNAPSHOT_22-June-2016T10-25-27
$ wget http://tinyurl.com/zom7pbw
$ tar -xvzf archappl_v0.0.1_SNAPSHOT_22-June-2016T10-25-27.tar.gz
$ rm archappl_v0.0.1_SNAPSHOT_22-June-2016T10-25-27.tar.gz
\end{lstlisting}

Nesta pasta estão os arquivos \textit{.war}  usados para lançar os módulos e
alguns \textit{scripts} sugeridos pelo desenvolvedor para auxiliar na
instalação.
	
  \item \label{item:appl} Crie uma pasta onde as instâncias \textit{Tomcat} de cada módulo serão
  armazenadas e, dentro dela, o arquivo \textit{lnls\_appliances.xml}. Tal pasta
  será chamada de \path{lnls-control-archiver_v0.0.1_22-June-2016T10-25-27}.

\begin{lstlisting}[keywordstyle=\ttfamily, style=nonumbers]
$ mkdir $INSTALL_DIR/lnls-control-archiver_v0.0.1_22-June-2016T10-25-27
$ cd INSTALL_DIR/lnls-control-archiver_v0.0.1_22-June-2016T10-25-27
$ nano lnls_appliances.xml
\end{lstlisting}

Copie as seguintes linhas para esse arquivo. Se a configuração do
\textit{Apache httpd} não foi realizada, então o endereço de
\texttt{<data\_retrieval\_url>} deve ser igual ao endereço de
\texttt{<retrieval\_url>}. Senão:

\begin{lstlisting}[keywordstyle=\ttfamily, style=nonumbers]
<appliances> <appliance>
     <identity>lnls_control_appliance</identity>
     <cluster_inetport>localhost:12000</cluster_inetport>
     <mgmt_url>http://localhost:11995/mgmt/bpl</mgmt_url>
     <engine_url>http://localhost:11996/engine/bpl</engine_url>
     <etl_url>http://localhost:11997/etl/bpl</etl_url>
     <retrieval_url>http://localhost:11998/retrieval/bpl</retrieval_url>
     <data_retrieval_url>http://localhost/lnls-archiver</data_retrieval_url>
   </appliance>
 </appliances>
\end{lstlisting} 

Esse arquivo especifica os endereços e portas de cada um dos quatro módulos, bem
como o nome da \textit{appliance} (\texttt{lnls\_control\_appliance}). Os
números das portas foram escolhidos aleatoriamente, mas, por sugestão do desenvolver,
deve-se adotar uma sequência crescente a partir da porta do módulo
\texttt{mgmt}, que, no nosso caso, vale 11995. É importante observar que
\texttt{localhost} deve ser substituído pelo endereço que será utilizado para
referenciar os \textit{servlets} remotamente.
\texttt{\(<\)data\_retrieval\_url\(>\)} é o endereço que deverá ser utilizado a
fim de enviar e obter requisões HTTP e JSON, conforme figura
\ref{fig:epics_archiver} e \texttt{\(<\)cluster\_inetport\(>\)} é a combinação
\texttt{TCPIP address:port} usado para a comunicação entre as
\textit{appliances}. Por fim, adicione as váriaveis de ambiente ao arquivo
\texttt{.bashrc}.

\begin{lstlisting}[keywordstyle=\ttfamily, style=nonumbers]
export ARCHAPPL_APPLIANCES=$INSTALL_DIR/lnls-control-archiver_v0.0.1_22-June-2016T10-25-27/lnls_appliances.xml
export ARCHAPPL_MYIDENTITY=lnls_control_appliance
\end{lstlisting}

O desenvolvedor sugere também trocar a porta padrão usada pelo \textit{Apache
Tomcat} pela porta usada pelo módulo \texttt{mgmt}, isto é, 11995, e comentar
as linhas sobre o uso do conector AJP. Para isso, siga a sequência abaixo.

\begin{lstlisting}[keywordstyle=\ttfamily, style=nonumbers]
$ nano $CATALINA_HOME/conf/server.xml
### Edite a porta de forma a obter
[...]
<Connector port="11995" protocol="HTTP/1.1"
               connectionTimeout="20000"
               redirectPort="8443" />
[...]
### Comente o trecho
<!-- Define an AJP 1.3 Connector on port 8009 
    <Connector port="8009" protocol="AJP/1.3" redirectPort="8443" /> -->
\end{lstlisting}

Execute o \textit{script} disponibilizado pelo desenvolvedor para criar as 4
instâncias do \textit{Apache Tomcat}.

\begin{lstlisting}[keywordstyle=\ttfamily, style=nonumbers]
$ source /home/user/.bashrc
$ cd $INSTALL_DIR/archappl_v0.0.1_SNAPSHOT_22-June-2016T10-25-27/install_scripts
$ ./deployMultipleTomcats.py $INSTALL_DIR/lnls-control-archiver_v0.0.1_22-June-2016T10-25-27
\end{lstlisting}

Serão criadas 4 pastas dentro de
\path{${INSTALL_DIR}/lnls-control-archiver_v0.0.1_22-June-2016T10-25-27}, uma
para cada módulo.

\item É possível também alteração as configurações de \textit{log}. O
desenvolvedor sugere que sejam executados os seguintes comandos:

\begin{lstlisting}[keywordstyle=\ttfamily, style=nonumbers]
$ nano $CATALINA_HOME/lib/log4j.properties

### Arquivo log4j.properties
# Set root logger level and its only appender to A1.
log4j.rootLogger=ERROR, A1
log4j.logger.config.org.epics.archiverappliance=INFO
log4j.logger.org.apache.http=ERROR

# A1 is set to be a DailyRollingFileAppender
log4j.appender.A1=org.apache.log4j.DailyRollingFileAppender
log4j.appender.A1.File=arch.log
log4j.appender.A1.DatePattern='.'yyyy-MM-dd

# A1 uses PatternLayout.
log4j.appender.A1.layout=org.apache.log4j.PatternLayout
log4j.appender.A1.layout.ConversionPattern=%-4r [%t] %-5p %c %x - %m%n
\end{lstlisting}

\item O próximo passo é definir o arquivo \texttt{policies.py}. Esse arquivo
especifica as regras de armazenagem que serão utilizadas pelo arquivador. O
arquivo fornecido pelo desenvolver define, por exemplo, três modalidades de
armazenamento, que foram exemplificadas na subseção \textbf{Introdução}. É
possível ajustar neste arquivo parâmetros que alteram a taxa de armazenamento de
uma PV dependendo de sua taxa de aquisição, por exemplo. Para um primeiro
instante vamos usar esse arquivo fornecido, que está disponível na pasta
\texttt{WEB-INF/classes} dentro do arquivo \texttt{mgmt.war} em
\texttt{\$INSTALL\_DIR/archappl\_v0.0.1\_SNAPSHOT\_22-June-2016T10-25-27}.
Copie-o para
\texttt{\$INSTALL\_DIR/lnls-control-archiver\_v0.0.1\_22-June-2016T10-25 -27} e
renomeie-o para \texttt{lnls\_policies.py}. Enfim, defina variáveis de ambiente
usadas por este arquivo.

\begin{lstlisting}[keywordstyle=\ttfamily, style=nonumbers]
export ARCHAPPL_POLICIES=$INSTALL_DIR/lnls-control-archiver_v0.0.1_22-June-2016T10-25-27/lnls_policies.py
export ARCHAPPL_SHORT_TERM_FOLDER=$INSTALL_DIR/lnls-archiver-storage/STS
export ARCHAPPL_MEDIUM_TERM_FOLDER=$INSTALL_DIR/lnls-archiver-storage/MTS
export ARCHAPPL_LONG_TERM_FOLDER=$INSTALL_DIR/lnls-archiver-storage/LTS
\end{lstlisting}

As três últimas variáveis devem ser corretamente configuradas de acordo com a
disponibilidade de equipamentos, por exemplo. No nosso caso, um único diretório
é usado para os três tipos de armazenamento, sendo ele \texttt{lnls-archiver-storage/}.

\item \label{mysql-install}A próxima etapa é a criação de tabelas usadas pelo
EPICS Archiver Appliance. Vamos criar uma base chamada \texttt{lnls\_appliance\_database} e o
usuário \texttt{lnls\_user}, com todos os direitos de acesso a ela. No nosso
caso, a senha do usuário é \texttt{controle}. 

\begin{lstlisting}[basicstyle=\fontsize{9}{13}\selectfont\ttfamily,keywordstyle=\ttfamily,
style=nonumbers] 
$ mysql -u root -p
>> CREATE DATABASE lnls_appliance_database;
>> GRANT ALL ON lnls_appliance_database.* TO 'lnls_user'@'%' IDENTIFIED BY
'controle';
\end{lstlisting}

Acesse o servidor \textit{mysql} com esse usuário e execute o \textit{script}
de criação das tabelas disponibilizado pelo desenvolvedor:

\begin{lstlisting}[basicstyle=\fontsize{9}{13}\selectfont\ttfamily,keywordstyle=\ttfamily,
style=nonumbers] 
$ mysql -u lnls_user -p
>> USE lnls_appliance_database
>> source $INSTALL_DIR/archappl_v0.0.1_SNAPSHOT_22-June-2016T10-25-27/install_scripts/archappl_mysql.sql
\end{lstlisting}

\item É necessário fazer o \textit{download} do conector \textit{MySQL} usado
pelo \textit{Apache Tomcat}.

\begin{lstlisting}[keywordstyle=\ttfamily, style=nonumbers]
$ wget http://tinyurl.com/jf3cadl
\end{lstlisting} 

Abra este arquivo e extraia \texttt{mysql-connector-java-5.1.38-bin.jar} para o
diretório \texttt{\$CATALINA\_HOME/lib/}. Abra o arquivo
\textit{conf/context.xml} e adicione dentro de \texttt{\(<\)Context\(>\)}:

\begin{lstlisting}[keywordstyle=\ttfamily, style=nonumbers]
$ nano $CATALINA_HOME/conf/context.xml

<Context ...>
	<Resource   name="jdbc/archappl"
	      auth="Container"
	      type="javax.sql.DataSource"
	      factory="org.apache.tomcat.jdbc.pool.DataSourceFactory"
	      username="lnls_user"
	      password="controle" 
	      testWhileIdle="true"
	      testOnBorrow="true"
	      testOnReturn="false"
	      validationQuery="SELECT 1"
	      validationInterval="30000"
	      timeBetweenEvictionRunsMillis="30000"
	      maxActive="10" 
	      minIdle="2" 
	      maxWait="10000" 
	      initialSize="2"
	      removeAbandonedTimeout="60"
	      removeAbandoned="true"
	      logAbandoned="true"
	      minEvictableIdleTimeMillis="30000" 
	      jmxEnabled="true"
	      driverClassName="com.mysql.jdbc.Driver"
	      url="jdbc:mysql://localhost:3306/lnls_appliance_database"
	 />
</Context>
\end{lstlisting}

Note que modificamos os campos \texttt{username}, \texttt{password} e
\texttt{url} de acordo com a instalação no item \ref{mysql-install}.

\item É necessário copiar e extrair todos os arquivos \textit{.war} presentes em
\path{archappl_v0.0.1_SNAPSHOT_22-June-2016T10-25-27} para os respectivos
diretórios \texttt{webapps/}. Para tal, definimos:

\begin{lstlisting}[keywordstyle=\ttfamily, style=nonumbers]
export DEPLOY_DIR=$INSTALL_DIR/lnls-control-archiver_v0.0.1_22-June-2016T10-25-27
export WARSRC_DIR=$INSTALL_DIR/archappl_v0.0.1_SNAPSHOT_22-June-2016T10-25-27
\end{lstlisting}

E executamos o \textit{script} \texttt{lnls-archiver-extract-wars.sh}, presente
no repositório do grupo, para extrair os arquivos \textit{.war}.


\item Enfim, os 4 \textit{Tomcats} são lançados através do serviço
\textit{systemd}, que pode ser obtido através do repositório \textit{git} do
grupo, conforme comando abaixo. Tal serviço deve ser \textit{instalado} do
diretório \path{/lib/systemd/system}.

\begin{lstlisting}[keywordstyle=\ttfamily, style=nonumbers]
$ sudo systemctl start lnls-archiver.service
\end{lstlisting}

\item O arquivador é acessado através da URL (\path{localhost} pode ser
substituído pelo endereço IP que referencia o \textit{host})

\begin{lstlisting}[keywordstyle=\ttfamily, style=nonumbers]
http://localhost:11995/mgmt/ui/index.html
\end{lstlisting}

\end{enumerate} 

\subsubsection{\textit{Debugging}}

A lista abaixo apresenta algumas situações que foram encontradas durante a
instalação.

\begin {enumerate}[i.] 
  \item Se após o reinício do arquivador todos os dados referentes às variáveis
  forem perdidos é necessário checar todos os arquivos \path{conf/context.xml}
  de cada um dos \textit{tomcats} e verificar que a configuração do servidor
  \textit{MySQL} foi realizada corretamente.

  \item Uma regra geral é verificar se todas as variáveis de ambiente foram
  corretamente configuradas.
\end{enumerate}

\subsection{Uso do CS Studio no monitoramento} \label{appliance-csstudio}

O \textit{CS Studio} pode ser usado para monitorar a \textit{appliance}. Para
isso, entre em \texttt{Edit \(>\) Preferences} e acesse o item \texttt{CSS
Applications \(>\) Trends \(>\) Data Browser}. No campo \textit{Archive Data
Server URLs}, adicione o endereço contido em \texttt{<data\_retrieval\_url>} do
arquivo configurado no item \ref{item:appl} da subseção \ref{sec:epicsappl},
substituindo o protocolo \texttt{http://} por \texttt{pbraw://}. Escreva
qualquer \textit{Server alias}. Na tabela \textit{Default Archive Data Sources},
adicione o mesmo endereço e aperte \textit{Ok} para salvar as alterações.

\vspace{12pt}

É necessário alterar a perspectiva do \textit{CS Studio}. Acesse
\texttt{Windows \(>\) Open Perspective} e escolha \textbf{Data Browser}. Na aba
\textit{Archive Search}, escreva a \textit{URL} configurada anteriormente e no
campo \textit{Pattern}, escreva o nome das variáveis arquivadas que deseja
monitorar. Por exemplo, se escrevermos \texttt{MTTemp*}, todas as variáveis
arquivadas com este início poderão ser acessadas. Clique com o botão direito na
variável desejada e acesse \texttt{Process Variable \(>\) Data Browser}.

\subsection{Acessando a \textit{appliance} com \textit{Python}}

A \textit{appliance} pode ser acessada através de requisições \textit{JSON}
realizadas por um módulo escrito em \textit{Python}, por exemplo. Abaixo, está
apresentada uma classe que foi escrita a fim de obter dados e informações de
variáveis arquivadas.

\vspace{12pt}

\begin{lstlisting}[language=Python]
import time
import urllib2
import json

class JsonRequester ():
    
    def __init__(self, data_retrieval_url, mgmt_url):
        self.data_retrieval_url = data_retrieval_url
        self.mgmt_url = mgmt_url
    
    def json_request_variables(self, variables_prefix):
        
        url_json = self.mgmt_url + 'bpl/getPVStatus?pv=' + variables_prefix
        req = urllib2.urlopen(url_json)
        data = json.load(req)
        return data
    
    def json_request_data(self, variable, from_date, to_date):    
        
        retrieval_url = self.data_retrieval_url + "/data/getData.json?"
        pv_name = ("pv=" + variable).replace(':', '%3A')
        to_date =   ("&to=" + time.strftime("%Y-%m-%dT%H:%M:%S",to_date) + 
        				".000Z").replace(':', '%3A')
        from_date = ("&from=" + time.strftime("%Y-%m-%dT%H:%M:%S",from_date) + 
        				".000Z").replace(':', '%3A')    
        url_json = retrieval_url + pv_name + from_date + to_date
        req = urllib2.urlopen(url_json)
        data = json.load(req)
        secs = [x['secs'] for x in data[0]['data']]
        vals = [x['val'] for x in data[0]['data']]
        return secs, vals
\end{lstlisting}

O método construtor recebe 2 \textit{strings} como parâmetros.
\textit{data\_retrieval\_url} e \textit{mgmt\_url} estão contidos no arquivo
\textit{lnls\_appliances.xml} e representam, respectivamente, os endereços dos
\textit{servlets} de obtenção de dados e gerenciamento da \textit{appliance}. A
primeira \textit{url} será usada para recuperar os dados e a segunda, para obter
as informações relativas às variáveis arquivadas.

\vspace{12pt}

O método \textit{json\_request\_variables} é responsável por retornar
informações de uma ou várias variáveis, cujo nome (no caso de uma pesquisa de
uma única variável) ou prefixo (parte comum ao nome de diversas variáveis) é
passado como parâmetro. Para tal, utiliza o método \texttt{getPVStatus}, que é
disponível no \textit{servlet} de gerenciamento e acessível via a \textit{url}
\texttt{mgmt\_url}. Esse método também recebe como parâmetro nomes ou prefixos
de variáveis, especificados após o trecho \texttt{pv=} na requisição
\textit{json}. Para recuperar todas as variáveis arquivadas por uma
\textit{appliance} que comecem com o prefixo \texttt{MBTemp}, por exemplo,
bastar utilizarmos \texttt{pv=MBTemp*} na requisição. Uma vez construída a
\textit{url}, é necessário utilizar as bibliotecas \textit{python}
\texttt{urllib2} e \texttt{json}, que realizam a comunicação com os
\textit{servlets}. 

\vspace{12pt}

O método \textit{json\_request\_data} retorna os dados arquivados para uma
determinada variável, passada como parâmetro. Além dela, essa função recebe dois
outros valores, sendo eles objetos do tipo \textit{time}, cuja implementação
reside no módulo \texttt{time}. Esses parâmetros representam, por sua vez, as
fronteiras do intervalo de tempo para o qual se deseja recuperar os dados, sendo
que \texttt{from\_date} é a data mais antiga e \texttt{to\_date}, a mais
recente. Se \texttt{to\_date} vale \texttt{None}, então o sistema o interpreta
como o tempo no qual a chamada da função foi feita. Os dados são recuperados
através do método \texttt{getData}, disponível no \textit{servlet}
\texttt{retrieval} da \textit{appliance}. Antes de realizarmos a requisição, é
necessário traduzir os objetos \textit{time} para o formato aceito pela
servidor. Sendo assim, utiliza-se o método \texttt{strftime} do módulo
\texttt{time} que retorna a representação \textit{string}, segundo a máscara
especificada (no nosso caso, \texttt{\%Y-\%m-\%dT\%H:\%M:\%S}), do objeto
passado como parâmetro. A requisição é, enfim, realizada através dos mesmos
métodos que foram usados na função anterior.

\vspace{12pt}

O servidor envia todos os dados, isto é, valores e datas do respectivo evento,
em único vetor de dicionários. Por este motivo, é necessário processarmos essa
estrutura antes de retorná-la ao programa que chamou o método. Os valores são
acessíveis pelo índice \texttt{val} e seu tipo depende da aplicação. A data dos
eventos relativos aos valores são recuperados pelo índice \texttt{secs} e são
números inteiros que contém o número de segundos desde uma data de referência
(1 de Janeiro de 1970) usada no \textit{servlet}. É necessário notar que as
datas retornadas estão em \textit{UTC}, portanto é exigido que tais
valores sejam convertidos para a o fuso local.

\vspace{12pt}

Uma interface gráfica foi implementada, usando os módulos \textit{Qt}, a fim de
testarmos a comunicação. Ela possui um gráfico, onde serão mostrados os dados
recuperados, uma caixa de opções, que possui todas as variáveis arquivadas na
\textit{appliance}, e componentes para seleção das datas de início e fim do
intervalo desejado. A figura \ref{fig:interface} representa o resultado da
implementação.

\FloatBarrier

\begin{figure}[h]
    
    \centering
    \includegraphics[scale=0.38]{image/screenshot-python}
    \caption {Interface \textit{Qt} implementada em \textit{python}.}
    \label{fig:interface} 
\end{figure} 

\FloatBarrier
