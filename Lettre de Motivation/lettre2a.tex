
\documentclass[12pt, a4paper]{article}


%=========================== PACKAGES =============================%
\usepackage[utf8]{inputenc}
\usepackage[hmargin=1.5cm,vmargin=1.5cm]{geometry}

%--- LANGUE--%
\usepackage[french]{babel}
%--- LANGUE--%

%--- FONT ----%
%\usepackage{helvet}
%\renewcommand{\familydefault}{\sfdefault}
%--- FONT ----%
%=========================== PACKAGES =============================%


%================== VARIABLES ======================%
\newcommand{\entreprise}{Gemalto }
\newcommand{\offreReference}{2014-6785}
\newcommand{\contactEntreprise}{Madame, Monsieur}
\newcommand{\contactEntrepriseGender}{Madame, Monsieur}
%===================================================%

\begin{document}

\pagestyle{empty} 

%======== PRENON ET NOM =========%
Gustavo CIOTTO PINTON
%======== PRENON ET NOM =========%

%======== ADRESSE ===============%
Résidence C.E.S.A.L. Supélec

1 rue Joliot-Curie 91192 Gif-sur-Yvette, France

(+33) 6 30 30 25 86	

\textit{gustavociotto@gmail.com}

\textit{fr.linkedin.com/in/gustavociotto}
%======== ADRESSE ===============%

\vspace{12pt}

\textbf{Objet:} Candidature aux postes de
références \textit{\textbf{\offreReference}}

%Références \textbf{[Ref 2015-02]}, \textbf{[Ref 2015-07]},
%\textbf{[Ref 2015-09]}, \textbf{[Ref 2015-10]} et \textbf{[Ref 2015-14]}.


\begin{flushright}
%======== VILLE ET DATE =========%
Faite à Gif-sur-Yvette, le \today
%======== VILLE ET DATE =========%

%======== ENTREPRISE ===============%

%%%%%%% CHANGER 
%\entreprise

%\vspace{12pt}

%A l'attention de \contactEntreprise
%======== ENTREPRISE ===============%
\end{flushright}

\contactEntrepriseGender ,

\vspace{12pt}

%======== LETTRE ===============%

Étant étudiant en dernière année dans le cadre de double diplôme en
ingénierie électrique à l’École Supérieure d’électricité, Supélec, je
suis à la recherche d'un stage de deux à six mois à partir de juillet.
%C'est au moyen de cette lettre que je vous communique mon intérêt notamment par
%les offres avec les références suivantes: \textbf{[Ref 2015-02]}, \textbf{[Ref
%2015-07]}, \textbf{[Ref 2015-09]}, \textbf{[Ref 2015-10]} et \textbf{[Ref
% 2015-14]}.
%
%
%C'est au moyen de cette lettre que je vous communique mes sincères prétentions
% à réaliser un stage lié au domaine d'architecture des systèmes informatiques -
% gestion des ressources telles comme la mémoire, traitement des données,
% programmation parallèle etc - chez \entreprise.
C'est au moyen de cette lettre que je vous communique mes sincères 
prétentions à réaliser un stage lié au domaine de programmation des systèmes
informatiques, particulièrement ceux dans lesquels je serai capable
d'apprimorer mes compétences au niveau de la programmation orientée à objets,
comme C++, et de bas niveau, particulièrement pour l'architecture ARM. L'offre
de référence \textit{\textbf{\offreReference}} corresponde bien à ces
caractéristiques et donc m'intéresse beaucoup.

\vspace{12pt}

L'informatique a été toujours au centre de mon projet professionnel, ce qui est
vérifié notamment par mon parcours académique. Avant de débuter mes études
universitaires au Brésil, j'ai réalisé un cours de nature technique à l'École
Technique de Campinas, dans lequel j'ai pu acquérir des connaissances très
solides par rapport à la programmation, soit celle orientée à objets (en Java)
soit celle fonctionnelle (en C), à la gestion de base de données (modèle
relationnel et langage SQL) et les principes de fonctionnement d'un réseau
informatique (réseaux TCP/IP). Ces connaissances m'ont bien permis d'avoir une
première expérience professionnelle chez une entreprise brésilienne de
développement d'applications mobiles, où j'étais responsable pour gérer les
bases de données. Ensuite, j'ai démarré mes études d'ingénierie informatique à
l'Université de Campinas, UNICAMP, qui m'ont mis en contact avec plusieurs
aspects théoriques de l’informatique, notamment la théorie derrière les 
structures de données, l'analyse qualitative des algorithmes (complexité,
utilisation de la mémoire etc.) et surtout la programmation de bas niveau et les
principaux composants de l'architecture des systèmes informatiques, dont le
domaine a énormément suscité mon intérêt, spécialement professionnel. Par
ailleurs, ce cours m'a également approché de nombreux principes de
l'ingénierie électrique, ce qui m'a permis de me candidater au programme de
double diplôme en ingénierie à Supélec. Cette dernière étape a joué un rôle
très important, une fois que j'ai pu consolider des concepts techniques
essentiels, comme le traitement du signal et l'automatique par exemple , mais
aussi avoir une expérience internationale très enrichie.

\vspace{12pt}

Obtenir des connaissances et des expériences est sûrement l'un des mes
principaux objets, mais il faut remarquer que je suis aussi engagé à
vous apporter des services de qualité avec efficacité et assiduité, tels comme
ceux qui m'ont fait devenir l'un des premiers élèves de toutes mes promos
et qui, surtout, soient au niveau des services fournis par \entreprise. Enfin,
travailler chez \entreprise serait un grand pas dans ma carrière, surtout au
niveau des expériences qui pourraient être acquises et principalement à la
réalisation d'un possible doctorat dans l'avenir, ce qui fait partie de mon
projet professionnel.

\vspace{12pt}

En espérant que ma candidature aura su retenir votre attention, je vous prie
d'agréer l'expression de mes sentiments.

\vspace{12pt}

\textbf{Gustavo CIOTTO PINTON}
%======== LETTRE ===============%


\end{document}
