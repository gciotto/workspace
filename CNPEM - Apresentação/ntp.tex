\section {Implementação de um servidor NTP}

%\subsection{Introdução}

\begin{frame}
\frametitle{Implementação de um servidor NTP}
\framesubtitle{Introdução}

\begin{itemize}
  \item Problema: como sincronizar todos os \textit{logs} e
  \textit{timestamps}?
  \item Receptores GPS fornecem:
  \begin{itemize}
    \item Horário no formato UTC.
    \item Pulsos 1PPS (\textit{pulse-per-second}): início de um novo
    segundo e precisão na ordem de nanossegundos.
  \end{itemize}
  \item NTP + Hora UTC + 1PPS = servidor de \textit{stratum 1}
  \vspace{8pt}
  \item 3 modelos comprados:
  \begin{itemize}
    \item \textit{Ultimate GPS Breakout} da \textit{Adafruit}.
    \item \textit{GPS Click} da \textit{MikroElektronika}. 
    \item \textit{CAM M8Q} da \textit{ublox}.
  \end{itemize} 
\end{itemize}

\end{frame}

% \begin{frame}
% \frametitle{Implementação de um servidor NTP}
% \framesubtitle{Instalação}
% 
% 
% \begin{itemize}
%   \item \textit{Ultimate GPS Breakout}:
%   \begin{figure}[h]
%     \centering
%     \includegraphics[width=0.80\textwidth]{image/adafruit_GPS}
%     \label{img:adafruit} 
%   \end{figure} 
%   \item \textit{GPS Click}:
%   \begin{figure}[h]
%     \centering
%     \includegraphics[width=0.80\textwidth]{image/mikroe}
%     \label{img:mikroe} 
%   \end{figure} 
%   \item Comparação:
%   \begin{figure}[h!]
%     \centering
%     \includegraphics[width=0.80\textwidth]{image/cliente-ntp}
%     \label{img:cliente_ntp} 
% \end{figure} 
% \end{itemize}
% 
% \end{frame}

%\subsection{Implementação de variáveis EPICS}
 
% \begin{frame}
% \frametitle{Implementação de um servidor NTP}
% \framesubtitle{Variáves EPICS - \textit{daemon NTPD}}
% 
% \begin{itemize}
%   \item Variáveis EPICS definidas para o \textit{daemon NTPD} :
%   
%   \vspace{12pt}
%   
% 	\begin{itemize}
% 	\item  \textcolor{red}{\textbf{\texttt{NTP:Leap}}: indicação de um eventual
% 	\textit{leap second} pendente.}
% 	\item \texttt{NTP:Stratum}: \textit{stratum} do servidor.
% 	\item \texttt{NTP:Refid}: ID de referência da fonte de sincronismo.
% 	\item \textcolor{red}{\textbf{\texttt{NTP:Offset}}: \textit{offset} entre o
% 	relógio do sistema e da fonte.}
% 	\item \textcolor{red}{\textbf{\texttt{NTP:Jitter}}: desvio padrão das medidas
% 	\textit{offset} mais recentes.}
% 	\item \texttt{NTP:Precision}: precisão do \textit{clock} do sistema em \textit{log2}.
% 	\item \texttt{NTP:Srcadr}: endereço IP do servidor. 			
% 	\item \texttt{NTP:Version}: versão e data de compilação.
% 	\item \textcolor{red}{\textbf{\texttt{NTP:Timestamp}}: diferença, em segundos,
% 	entre a data atual do servidor e a \textit{unix epoch} (1 de Janeiro de 1970).}
% 	\end{itemize}	    
% \end{itemize}	
% 
% \end{frame}
% 
% \begin{frame}
% \frametitle{Implementação de um servidor NTP}
% \framesubtitle{Variáves EPICS - \textit{daemon GPSD}}
% 
% \begin{itemize}
%   \item Variáveis EPICS definidas para o \textit{daemon GPSD} :
%   \vspace{12pt}
% 	\begin{itemize}
% 	\item \textcolor{red}{\textbf{\texttt{GPS:Fix}}: indicação do tipo de
% 	\textit{fix} (\textit{no fix}, 2D ou 3D \textit{fix}).}
% 	\item \texttt{GPS:Latitude}: latitude medida pelo GPS.
% 	\item \texttt{GPS:Longitude}: longitude medida pelo GPS.
% 	\item \texttt{GPS:Altitude}: altitude medida pelo GPS.
% 	\item \textcolor{red}{\textbf{\texttt{GPS:Sattelites}}: \textit{string}
% 	contendo a identificação dos satélites usados para o \textit{fix}.}
% 	\item \textcolor{red}{\textbf{\texttt{GPS:Timestamp}}: \textit{timestamp}
% 	fornecido, análogo à mesma variável do servidor NTP.}
% 	\end{itemize}	    
% \end{itemize}
% \end{frame}

\begin{frame}

\frametitle{Implementação de um servidor NTP}
\framesubtitle{Variáves EPICS}

\begin{itemize}
  \item Variáveis EPICS definidas para o \textit{daemon NTPD} :

	\begin{itemize}
% 	\item  \texttt{NTP:Leap}: indicação de um eventual
% 	\textit{leap second} pendente.
	\item \texttt{NTP:Offset}: \textit{offset} entre o
	relógio do sistema e da fonte.
	\item \texttt{NTP:Jitter}: desvio padrão das medidas
	\textit{offset} mais recentes.
	\item \texttt{NTP:Timestamp}.
	\item \ldots
	\end{itemize}
  \item Variáveis EPICS definidas para o \textit{daemon GPSD} :
  
	\begin{itemize}
	\item \texttt{GPS:Fix}: indicação do tipo de
	\textit{fix} (\textit{no fix}, 2D ou 3D \textit{fix}).
	\item \texttt{GPS:Sattelites}: satélites usados para o \textit{fix}.
	\item \texttt{GPS:Timestamp}.
	\item \ldots
	\end{itemize}
\end{itemize}
\end{frame}

\begin{frame}
\frametitle{Implementação de um servidor NTP}
\framesubtitle{Interface gráfica} 
\begin{columns}
\begin{column}{.5\textwidth}
\begin{figure}[h]
    \centering
    \includegraphics[width=0.85\textwidth]{image/gps} \\
    \includegraphics[width=0.85\textwidth]{image/gps-ublox}
    \caption {Receptores GPS.}
    \label{img:gpses} 
\end{figure}
\end{column} 
\begin{column}{.5\textwidth}
\begin{figure}[h]
    \centering
    \includegraphics[width=\textwidth]{image/epics-opi-ntpgps}
    \caption {Interface construída para visualização das
    \textit{PVs}.}
    \label{img:ntp-opi} 
\end{figure}
\end{column} 
\end{columns}

\end{frame}