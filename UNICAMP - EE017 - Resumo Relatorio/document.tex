\documentclass[12pt, a4paper]{article}


%=========================== PACKAGES =============================%
\usepackage[utf8]{inputenc}
\usepackage[hmargin=1.75cm,vmargin=1.5cm]{geometry}

%--- LANGUE--%
\usepackage[brazil]{babel}
\usepackage{hyperref}
%--- LANGUE--%

%--- FONT ----%
% \usepackage{helvet}
% \renewcommand{\familydefault}{\sfdefault}
%--- FONT ----%
%=========================== PACKAGES =============================%

\begin{document}

\pagestyle{empty} 

%======== PRENON ET NOM =========%
Gustavo CIOTTO PINTON

RA 117136 - Engenharia de Computação
%======== PRENON ET NOM =========%

%======== ADRESSE ===============%
Rua Voluntário Amador Lourenço, 57

Valinhos, São Paulo, Brasil CEP 13271-393

+55 19 99898-5633

\url{gustavociotto@gmail.com}

\url{http://gciotto.github.io/workspace/}


\begin{center}
\textbf{Resumo do Relatório}
\end{center}

O relatório é composto por 5 seções, correspondendo cada uma a um projeto
realizado durante o estágio. O primeiro explica as tarefas desenvolvidas para a
implementação de um servidor NTP de \textit{stratum} 1 a partir de um receptor
GPS (3 modelos foram utilizados) e a \textit{BeagleBone Black}. Além disso,
especificamos também as variáveis de leitura obtidas deste sistema. O
intuito do projeto 2 foi, por sua vez, desenvolver um sistema de sincronismo via
Ethernet a partir da \textit{Beagle}. Neste projeto, desenvolveu-se módulos para
o \textit{kernel} do Linux e uma aplicação para o módulo PRU
(\textit{programmable real-time unit}) da \textit{Beagle} e comparou-se as
diferenças entre os dois sistemas. O terceiro projeto consistiu na manutenção do
principal programa do sistema de controle do UVX, chamado de PROSAC. Nesta
etapa, implementou-se aplicações cliente em diferentes linguagens de
programação, como Java e C, para, respectivamente, PCs de mesa e para o
\textit{kit} STM32F746G \textit{discovery kit}.  Ainda neste tema, participei da
elaboração de um artigo, lançada na conferência \textit{Personal Computers and
Particle Accelerator Controls} de 2016, cujo intuito foi adaptar o PROSAC para a
\textit{BeagleBone Black}. No quarto projeto, instalei e adicionei código ao
arquivador de variáveis, chamado de  \textit{EPICS Archiver Appliances}.
Adicionei trechos ao programa em Java para verificar se um usuário foi
corretamente autenticado. Enfim, no quinto projeto, instalei e configurei o
servidor de alarmes \textit{BEAST}, que está sendo utilizado atualmente no
sistema de controle do \textit{UVX}.

\vspace{12pt}

É importante observar que estes não foram os únicos projetos desenvolvidos
durante o estágio. Os menores não foram reportados, visto que o CNPEM impôs o
máximo de 20 páginas no relatório. Dessa forma, não pude escrever sobre todos
eles.

\end{document}
