%%This is a very basic article template.
%%There is just one section and two subsections.
\documentclass{article}

\usepackage[utf8]{inputenc}
\usepackage[hmargin=3.0cm,vmargin=3.0cm]{geometry}
\usepackage[portuguese]{babel}
\usepackage{graphicx}

\begin{document}

\begin{titlepage}
\vspace*{.28\textheight}
\begin{center}
%
\begin{figure}[h]
    \centering
    \includegraphics[scale=0.12]{images/LogoSupelec}
\end{figure}
%
\vspace*{10pt}
%\text{ }\\[7 cm]
\textbf{\LARGE{VALORISATION D'UN SUJET SCIENTIFIQUE}}\\[2 cm]
Gustavo \textbf{CIOTTO PINTON}\\[1 cm]
Supélec, 2015

\end{center}
\end{titlepage}

\newpage

\section{Introdução}

A atividade de pesquisa é um campo fundamental a qualquer área da indústria,
sendo assim principal fonte de inovação de uma empresa. Representando um ramo
tão importante de uma economia, tal atividade deve ser apresentada de maneira a
convencer a sociedade de sua importância. É nesse contexto que a eletiva
''valorisation d'un sujet scientifique'' nos propõe maneiras e técnicas para
atrair a atenção a nosso tema.

\vspace{12pt}

A fim de colocar em prática tais conceitos, nada melhor do que conhecer as
experiências de um profissional que já atua nesse setor e que, consequentemente,
já está acostumado a participar de conferências e a escrever textos de natureza
científica. O objetivo deste documento é assim de explorar tais experiências.  

\vspace{12pt}

O assunto escolhido a este trabalho foi um tema que me atrai muito, que é a área
da microeletrônica digital, que atualmente uma das áreas mais importantes da
indústria, dado o sucesso dos aparelhos eletrônicos como tablets e aparelhos
celulares. A problemática do sujeto tratado neste documento consiste no diálogo
e adaptação entre as diversas partes analógicas e digitais de um determinado
circuito, que envolvem evidentemente conceitos matemáticos completamente
distintos. Parte do problema reside no fato de que atualmente existem poucos
simuladores e aqueles que existem são muito complexos e muito específicos para
determinada aplicação.

\vspace{12pt}

Considerando este cenário que o doutorando Daniel C. CAFÉ, do departamento de
Informatique da Supelec, entrevistado por mim, encontrou uma oportunidade de
tese.  Seus estudos têm o objetivo, de maneira geral, de desenvolver uma
adaptação semântica entre os dois domínios do circuito para que, assim, novas 
ferramentas permitindo simular circuitos digitais e analógicos sob vários
pontos, como por exemplo, a influência do tipo de bus de comunicação escolhido,
possam ser criadas. Essas aplicações possuem evidentemente uma grande
importância já que elas podem ser utilizadas independentemente da natureza e
aplicação do circuito, uma vez que elas introduzem conceitos genéricos.


\newpage


\section{A tese de doutorado}

A parte de simulação de sistemas mistos, isto é, sistemas contendo partes
analógicas e digitais, é problemática pelo fato de que os modelos matemáticos
utilizados por cada um deles são essencialmente distintos. Pelo lado analógico,
existem os clássicos algoritmos resolvedores de equações diferenciais que
calculam a cada instante de tempo um resultado válido. Na parte digital, por
outro lado, eventos discretos são o objeto de manipulação do circuito. Se
pegarmos o exemplo de um circuito contendo diversas portas lógicas ou
flip-flops, o resultado da alteração de seus estados e sua saída serão
modificados logo depois de um evento bem definido, geralmente representado por
um pulso de clock (somado com o intervalo de propagação dos sinais).
Nós temos assim duas maneiras completamente diferentes de realizar os cálculos,
o que impõe problemas consideráveis na construção de circuitos mistos, já que,
de um lado, nós podemos ativar o lado digital toda vez que um cálculo da parte
analógica for realizado ou, ao contrário, ativar a parte analógico somente
quando um evento do lado digital for gerado.

\vspace{12pt}

A fim de resolver essa problemática, nós fazemos uso de uma teoria da computação
chamada de adaptação semântica, que consiste justamente à pegar os conceitos de
um domínio e adaptá-los a um outro qualquer em três eixos: tempo, dados e
controle. No eixo do tempo, por exemplo, imaginemos que determinado sistema seja
responsável por converter um tempo em segundos para uma rotação, isto é,
desejamos medir o tempo de uma rotação de uma engrenagem. De um lado, temos o
tempo em segundo e, de outro, em ângulos. Em relação à adaptação de dados, a
conversão é simples: se nós quisermos traduzir um medida analógica em volts para
um nível digital binário, é necessário introduzirmos um sentido à tradução, isto
é, a partir de qual voltagem consideramos a medida como 1 ou como 0?
Evidentemente isto depende das características da aplicação, ou em outras
palavras, da semântica de adaptação da aplicação. Finalmente, o último e mais
complicado eixo, o de controle, determina como os componentes serão ligados no
circuito e como eles se comunicarão. Por exemplo, nós podemos definir que um fio
representa um barramento de dados ou uma fifo, se estivermos modelisando um
software. Podemos especificar ainda o protocolo seguido para a transmissão:
ethernet, tcp/ip ou qualquer outro. De maneira geral, a adaptação deste eixo
leva em conta o significado de cada componente, ou no nosso caso, o fio.

\vspace{12pt}

Esses três eixos formam assim o centro da teoria de adaptação semântica, que, à
princípio, foi desenvolvida para o domínio da computação. O objetivo da tese
desenvolvida pelo meu entrevistado é justamente aplicar esses conceitos ao mundo
da microeletrônica, especificamente para o caso de circuitos mistos.

\vspace{12pt}

Segundo ele, seu primeiro ano de doutorado foi utilizado inteiramente para o
estudo das refêrencias existentes na área, isto é, as ferramentas já
implementadas e os pesquisadores que também se interessavam pelo domínio. Nesta
atividade, Daniel encontrou três simuladores que resolvem o problema: o
primeiro é chamado Modélix e é utilizado na Supélec, o segundo, Pitoremi, foi
desenvolvido na universidade de Berkeley nos Estados Unidos e o terceiro,
chamado de SystemC. Todos eles se baseiam sobre conceitos relativamente
similares, mas lidam com o problema de uma maneira diferente. As estudando
profundamente, o doutorando sentiu a necessidade de utilizar somente os pontos
mais pertinentes de cada uma dessas soluções para criar, enfim, sua ferramenta. 

\vspace{12pt}

De maneira geral, essa solução consiste à construir o diagrama do circuito em
alto nível, isto é, em forma de blocos e anotar em cada um de seus componentes a
sua função semântica (em outras palavras, o seu significado). É necessário
também dizer que só isso não é suficiente, já que ainda é preciso anotar a
semântica de adaptação, que consiste a definir o significado dos fios que ligam
os diversos blocos do sistema. Dessa forma, o simulador conhece claramente o que
cada componente significa e qual é a sua função. Por exemplo, nós podemos
especificar quais adaptações o simulador terá que realizar se quisermos
fazer uma adaptação do análogico para o digital ou do software para o hardware.

\vspace{12pt}

Então, toda essa capacidade de especificação se mostra como uma ferramenta
muito poderosa porque ela evita o uso de vários simuladores distintos, uma vez
que a solução proposta é capaz de se adaptar e simular a comunicação entre os
diversos tipos de sistemas (mêcanico, életrico, analógico), e, consequentemente,
evita também a realização dos cálculos muito pesados inerentes de cada um desses
sistemas. Dessa forma, a simulação de sistemas muito complexos, como por
exemplo, a simulação inteira de aviões ou carros, onde sistemas de diversos
tipos se comunicam continuamente, se torna mais simples e menos custosa. Enfim,
esse é o segredo que torna o valor dessa pesquisa tão elevado e atrativo às
diversas empresas de engenharia.

\section{Valorização}

Em relação ao processo de escrita de artigo científico, Daniel comentou que uma
das dificuldades é justamente explicar de uma maneira eficiente um o domínio de
um pesquisador sobre um determinado tema. Para tal, ele afirmou que um artigo
científico sempre segue um padrão, sendo constituido de uma introdução, que
explicará brevemente o seu problema, seguida de uma seção que resumirá o arte da
arte atual, isto é, como as outras pessoas do mundo resolveram ou estão tentando
resolver o problema, as ferramentas utilizadas, desenvolvimento teórico e
matemático e, por fim, uma conclusão com algumas perpectivas.

\vspace{12pt}

O doutorando ainda me explicou que a conclusão é a parte mais importante de um
artigo, já que é essa seção que determinará se ele será posteriormente lido ou
não.Em outras palavras, a conclusão deve chamar a atenção do leitor a sua
solução de forma objetiva e clara. Se a conclusão atendeu às expectativas, a
próxima seção a ser lida é a introdução. Nela, podemos encontrar uma breve
descrição do problema a ser resolvido. Em seguida, as demais seções são lidas
dependendo da necessidade do pesquisador.

\vspace{12pt}

Existem diversas formas de publicar um artigo. Nós, como engenheiros, temos dois
alvos principais: o primeiro trata-se das revistas da IEEE e o segundo, da ACM
(mais direcionadas às comunidades que lidam com problemas na área de
computação). Na IEEE, eles utilizam um padrão de tal forma que o texto fique
separado em duas colunas e que a bibliografia seja numerotada, enquanto que na
ACM os artigos publicados são escritos em apenas uma coluna com a presença de
mais espaços, o que permite uma melhor utilização de imagens por exemplo. 

\vspace{12pt}

O principal objetivo é então publicar nos journals, que são realizados apenas
uma vez a cada dois ou três anos. Neles, estão presentes somente as melhores
pesquisas e as publicações são mais densas, podendo possuir mais de 20 páginas
cada uma, já que o nível de detalhamento deve ser alto. Uma outra alternativa é
a publicação de artigos científicos de conferência, apresentada de uma forma
mais resumida e rápida (aproximadamente 8 páginas) e seguida de uma exposição
oral realizada por outros profissionais da área. Neste quesito, ele acrescentou
que o objetivo destes outros profissionais é de te desafiar a fim de testar se
a sua solução é realmente válida. Finalmente, o artigo passa por uma revisão
final. Se tudo está de acordo, o artigo é publicado nas chamadas transactions e
todos aqueles inscritos na comunidade podem acessá-lo. Como complemento, o
pesquisador pode escrever pequenas cartas, acrescentando pequenos trechos a sua
pesquisa (novas ideais, sugestões resultados etc).

\vspace{12pt}

Um outro aspecto muito importante a ser levado em conta também é o tipo de
financiamento de um doutorado. Existem basicamente duas opções. A primeira,
chamada CIFRE, consiste em uma colaboração com uma empresa. Esse tipo de acordo
é muito visado, uma vez que o pesquisador tem muitas chances de continuar dentro
da empresa mesmo depois do fim de sua tese, já que processo de implementação da
solução pode levar muitos anos. A segunda forma, chamada de CNRS, trata-se de um
financiamento dado pelo governo e pela escola.

\newpage

\section{Entrevista}

Na opinião do meu entrevistado, se dirigir à pesquisa ou não depende da vocação
de cada um, isto é, você pode resolver um problema utilizando um método que não
seja necessariamente um método científico. Como exemplo para explicar essa
afirmação, ele citou os casos de ''engenharia de garagem'', típicos da cultura
norte-americana, em que a pessoa resolve um determinado problema sem recorrer
aos métodos utilizados na academia. Ainda que essa pessoa tenha achado uma
solução a seu problema, ela pode não ser a mais eficiente, já que nenhuma outra 
refêrencia que já tinha resolvido a mesma situação foi procurada. Esse
comportamento se revela frequentemente também no meio industrial, uma vez que,
quando um problema aparece, o engenheiro tenta resovê-lo através das
ferramentas ao seu alcance e se esquece de uma etapa muito importante da
métodologia científica, que é a procura de refêrencias já existentes. Ainda
segundo Daniel, muitas vezes a solução reside em outros domínios de
atividade e comunidades (ele mesmo é exemplo disso, já que ele teve que buscar
a solução na área de informática para seu problema de microelêtronica). Assim,
ele conclui que a grande vantagem do doutorado é justamente a capacidade de
abrir a mente a outros domínios: se eu não estou resolvendo o problema ou se o
resolvo mal, então eu vou procurar a solução em alguma outra área.

\vspace{12pt}

Ainda neste tema, Daniel comentou de uma confêrencia em que ele participou
no Brasil onde o pricipal assunto discutido era exatamente essa problemática de
encontrar uma solução a um determinado problema. Segundo ele, muitas pessoas ao
redor do planeta possuem as mesmas ideias, mas o que realmente as diferencia é a
capacidade de realizar e sintetizar todas os conceitos em uma solução concreta.
Em outras palavras, não basta ter as ideias, é preciso também ter o conhecimento
técnico. Enfim, ele concluiu que essa característica é fundamental para o
fortalecimento e crescimento de uma empresa: capacidade de fusão de um carater
inovador com um conhecimento técnico que permita a implementação das ideias.

\vspace{12pt}

Ainda segundo meu entrevistado, a metodologia industrial é completamente
diferente daquela científica. No meio industrial, quando se deparamos com um
problema, a primeira ação é procurar quem vende a solução. Neste caso, não há um
processo de pesquisa ou estudo profundo do problema. Por outro lado, no meio
acadêmico, geralmente os problemas ainda não possuem um resposta, então a única
de forma de resolvê-los é através da metodologia científica: olhar o estado da
arte, comparar o seu problema com diversas outras pessoas do mundo e propor a
solução. É justamente nesse aspecto de criação de soluções que um pesquisador
pode se destacar em relação aos demais, à medida que ele consegue extender seus
conhecimentos a diversas áreas e publicar mais papers relevantes à comunidade.
Desse modo, ele torna-se mais procurado para a apresentação de conferências e
seminários e para novos projetos.

\vspace{12pt}

Quando perguntado sobre os motivos de realizar o doutorado em um país diferente
daquele de sua origem, ele me disse que essa era a sua única opção, uma vez que
o domínio da microeletrônia é ainda muito mal desenvolvido no Brasil, que é onde
ele nasceu. Ele ainda citou alguns exemplos de fábricas de chips no exterior que
são totalmente automatizadas, enquanto que, no Brasil, os profissionais são
forçados a fabricá-los quase que manualmente.

\vspace{12pt}
 
Em relação ao percurso profissional do entrevistado, antes de começar o seu
doutorado, ele já havia tido uma experiência em uma empresa dentro do ramo
de microeletrôncia, que, segundo ele, foi muito importante. Isso porque essa
atividade já havia lhe dado um conhecimento prático muito precioso e útil à
pesquisa. Resumindo, para Daniel, uma experiência profissional antes de um
doutorado é muito válida, já que é lá que o pesquisador vai estar em contato com
os verdadeiros problemas práticos e não com os puramente teóricos, como acontece
frequentemente na academia. Ele acrescentou ainda que uma experiência
internacional é fundamental para o enriquecimento do seu conhecimento, já que
você tem a oportunidade de entrar em contato com outras maneiras de resolução de
um determinado problema.
 
\vspace{12pt}

Enfim, eu questionei sobre a importância de um doutorado para conseguir um
emprego nas principais empresas de microeletrônica atuais e a resposta foi que,
na sua opinião, uma tese é muito importante já que ela revela o cárater inovador
e a capacidade de propor do pesquisador, duas características essenciais para o
crescimento de qualquer empresa.

\section{Conclusão}

A atividade de pesquisa possui um carater muito importante, já que é ela que
garante o crescimento e notoriedade de uma organização através da inovação.
Sendo assim, o sujeito científico torna-se um aspecto essential à apresentação
de uma nova solução a determinada comunidade e deve ser feita respeitando
os respectivos padrões impostos. Por exemplo, nós, como engenheiros,
direcionamos nossas publicações na grande maioria à IEEE e à ACM e, por isso,
devemos ter certeza que nosso texto atende às diversas imposições dessas
organizações. 

\vspace{12pt}

Ainda em relação à apresentação de um sujeito científico, devemos considerar
também o aspecto oral. Para isso, as aulas realizadas durante a sequência foram
muito úteis, uma vez que os palestrantes foram capazes de demonstrar através de
exercícios práticos os principais métodos que são utilizados atualmente para se
vender uma solução. Dentro desses métodos, destacam-se a utilização de pequenas
questões (accroches) e de exemplos concretos que causem a reflexão dos
interlocutores. É preciso remarcar igualmente que devemos sempre levar em conta
as características das pessoas a quem estamos falando, isto é, sejam eles
investidores ou consumidores finais, temos que ser capazes de transmitir nossas
idéias de maneira a capturar suas atenções.

\vspace{12pt}

Em relação à entrevista com o doutourando, tive a sensação que ela me esclarece
muito pontos em relação à atividade de pesquisa. O primeiro aspecto que eu não
tinha muito certeza era sobre como um profissional decide que ele quer um
doutorado. Antes, eu pensava que esse tipo de atividade era completamente
separada à atividade industrial, o que não verdade. As duas áreas se
complementam e, mais que isso, dependem uma da outra. Por exemplo, o meu
entrevistado só percebeu a necessidade de um doutorado depois de uma experiência na área
industrial. Sem dúvidas, atualmente um doutorado faz parte do meu projeto
profissional a partir de agora.

\end{document}
