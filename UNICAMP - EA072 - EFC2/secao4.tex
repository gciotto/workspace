\section{Sistema de recomendação empregando \textit{k}-NN}

\subsection {Abordagens \textit{User-based} e \textit{Item-based}}

De acordo com o \textit{paper} \textit{State-of-the-art Recommender Systems},
sistemas de recomendação que utilizam a abordagem \textit{user-based} exploram
as similaridades entre usuários que possuam características em comum para
predizer uma classificação que um usuário faria potencialmente para um item.
Sendo assim, define-se o conjunto \(\mathbb{U}\) formado pelos usuários , o
conjunto \(\mathbb{I}\) formado pelos itens e  o conjunto \(\mathbb{R}\) que
possui todas as classificações dadas.  Uma classificação \(r_{ui}\) é dada pelo
usuário \(u\in \mathbb{U}\) para o item \(i\in \mathbb{I}\)  e cada usuário
\(u\) possui um conjunto de itens avaliados \(\mathbb{S}_u \subseteq
\mathbb{I}\). A predição da classificação que um usuário \(a\in\mathbb{U}\)
faria para o item \(i \notin \mathbb{S}_a\) e que considere \(K\) elementos do
conjunto \(\mathbb{T}_a\), que contém \(K\) outros usuários cujas
preferências sejam as mais parecidas com as de \(a\), é então:

\begin{equation}
p_{ai} = \frac{\sum_{\{ u\in \mathbb{T}_a | i\in \mathbb{S}_u \}} \left(
sim(a,u) * r_{ui} \right)}{\sum_{\{ u\in \mathbb{T}_a | i\in \mathbb{S}_u \}}
\left| sim(a,u)\right|}
\end{equation}

ou

\begin{equation}
p_{ai} = \bar{r_a} + \frac{\sum_{\{ u\in \mathbb{T}_a | i\in \mathbb{S}_u \}}
\left( sim(a,u) * (r_{ui} - \bar{r_u})\right)}{\sum_{\{ u\in \mathbb{T}_a | i\in
\mathbb{S}_u \}} \left| sim(a,u)\right|}
\end{equation}

em que \(sim(a,u)\) mede o grau de similaridade entre \(a\) e \(u\) e
\(\bar{r_v}\) é a média das classificações de um usuário \(v\). A segunda
expressão leva em consideração as diferenças de intensidade entre os usuários
(por exemplo, em um sistema de 0 a 5, para \(u\), \textit{muito bom} pode ser
3.5, mas, para \(a\), isso é 5) Essas equações nos dizem, portanto, que a
predição realizada se aproximará da classificação feita pelo usuário \(u\) cujas
características são as mais parecidas em relação a aquelas de \(a\).

\vspace{12pt}

A abordagem \textit{item-based}, por sua vez, utiliza os \textit{ratings} de
itens \(j\) já classificados pelo usuário para predizer a possível classificação
que o respectivo usuário daria a um item \(i\), dado o grau de semelhança entre este
item não avaliado e aqueles avaliados. Define-se também a vizinhança
\(\mathbb{T}_i\), contendo \(K\) itens próximos de \(i\) em termos de
características. Assim como no caso anterior, essa predição \(p_{ai}\) pode ser
realizada de duas maneiras:


\begin{equation}
p_{ai} = \frac{\sum_{\{ j\in \mathbb{S}_a \cap \mathbb{T}_i \}} \left(
sim(i,j) * r_{aj} \right)}{\sum_{\{ j\in \mathbb{S}_a \cap \mathbb{T}_i \}}
\left| sim(i,j)\right|}
\end{equation}

ou

\begin{equation}
p_{ai} = \bar{r_i} + \frac{\sum_{\{ j\in \mathbb{S}_a \cap \mathbb{T}_i \}}
\left( sim(i,j) * (r_{aj} - \bar{r_j})\right)}{\sum_{\{ j\in \mathbb{S}_a \cap
\mathbb{T}_i \}} \left| sim(i,j)\right|}
\end{equation}

em que \(sim(i,j)\) mede o grau de similaridade entre os itens \(i\) e \(j\) e
\(\bar{r_k}\) é a média das classificações feitas para o item \(k\). Como na
abordagem anterior, considera-se que itens muito parecidos terão
\textit{ratings} também muito parecidos.

\subsection{Exemplo de Aplicação}
