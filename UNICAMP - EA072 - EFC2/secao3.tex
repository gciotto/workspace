\section{Controle Nebuloso e Robótica Evolutiva}

\setcounter{subsection}{-1}
\subsection{Definição dos Consequentes }

O primeiro passo no controle do robô é a definição dos consequentes das regras
que serão levadas em conta. Considerando que o robô dispõe de 3 sensores
(\(d_1\), \(d_2\) e \(d_3\)) e que as funções de pertinência associadas às
medidas realizadas por eles estão representadas na seção 3 do enunciado, sugestões de consequentes estão
representadas na tabela \ref{tab:consequentes}, logo abaixo. As regras são da
forma \textbf{SE (\(d_1\) É x) E (\(d_2\) É y) E (\(d_3\) É z) ENTÃO
(\(\Delta \Theta\) É w)}.

				\begin{table}[H]
					\centering
					\caption{\label{tab:consequentes} Regras que controlarão o robô.}
					\setlength\tabcolsep{4pt}
					\begin{minipage}{0.30\textwidth}
					    \centering
						\begin{tabular}{| c | c | c | c |} 
						\hline
						\(d_1\) & \(d_2\) & \(d_3\) & \(\Delta \Theta\) \\ \hline
						P & MP & P & MN  \\ \hline
						P & MP & M & MN \\ \hline
						P & MP & G & MN \\ \hline
						P & PP & P & Z  \\ \hline
						P & PP & M & MN \\ \hline
						P & PP & G & MN \\ \hline
						P & PG & P & Z  \\ \hline
						P & PG & M & MN \\ \hline
						P & PG & G & MN \\ \hline
						P & MG & P & Z  \\ \hline
						P & MG & M & PN \\ \hline
						P & MG & G & MN \\ \hline
						\end{tabular}	
					\end{minipage}	      
					\begin{minipage}{0.30\textwidth}
					    \centering
						\begin{tabular}{| c | c | c | c |} 
						\hline
						\(d_1\) & \(d_2\) & \(d_3\) & \(\Delta \Theta\) \\ \hline
						M & MP & P & PP \\ \hline
						M & MP & M & MP \\ \hline
						M & MP & G & MN \\ \hline
						M & PP & P & PP \\ \hline
						M & PP & M & Z  \\ \hline
						M & PP & G & PN \\ \hline
						M & PG & P & PP \\ \hline
						M & PG & M & Z  \\ \hline
						M & PG & G & PN \\ \hline
						M & MG & P & PP \\ \hline
						M & MG & M & Z  \\ \hline
						M & MG & G & PN \\ \hline
						\end{tabular}	
					\end{minipage}	
					\begin{minipage}{0.30\textwidth}
					    \centering
						\begin{tabular}{| c | c | c | c |} 
						\hline
						\(d_1\) & \(d_2\) & \(d_3\) & \(\Delta \Theta\) \\ \hline
						G & MP & P & MP \\ \hline
						G & MP & M & MP \\ \hline
						G & MP & G & MP \\ \hline
						G & PP & P & MP \\ \hline
						G & PP & M & MP \\ \hline
						G & PP & G & MP  \\ \hline 	  	 
						G & PG & P & MP \\ \hline
						G & PG & M & PP \\ \hline
						G & PG & G & Z  \\ \hline
						G & MG & P & MP \\ \hline
						G & MG & M & PP \\ \hline
						G & MG & G & Z  \\ \hline
						\end{tabular}	
					\end{minipage}	      
			    \end{table}
	\FloatBarrier
	
	\subsection{Controlando o Robô}
	
	Uma vez definidas as regras que deverão ser seguidas, o próximo passo é então
	simular o comportamento do robô através do \textit{software} MATLAB. Para isso,
	5 funções foram escritas. A primeira, chamada \texttt{trap\_pertinencia},
	recebe 6 parâmetros, sendo eles a coordenada \(x\), as informações \(a\), \(b\), \(c\)
	e \(d\) referentes os trapézios que compõem as funções de pertinência e o valor
	a ser considerado caso \(x\) esteja fora do intervalo \(\left[a, d\right]\), e
	retorna o respectivo valor da função de pertinência. O programa
	\ref{lst:trap_pertinencia} abaixo possui a implementação da respectiva
	função.
	
	\lstinputlisting [language=Matlab, caption={ \texttt{trap\_pertinencia.m} -
	Avalia pertinência de um regra em função de \(x\).},
	label={lst:trap_pertinencia}] {fuzzy/trap_pertinencia.m}
	
	\vspace{12pt}
	
	A segunda função, chamada de \texttt{get\_D1\_D3\_Rule} e representada pelo
	programa \ref{lst:d1d3}, calcula a pertinência de cada um dos possíveis estados
	referentes aos sensores \(D_1\) e \(D_3\).	Essa função recebe como parâmetro a
	distância medida pelo sensor e retorna um vetor com três valores contendo a
	pertinência para os estados \textbf{P}, \textbf{M} e \textbf{G}. 
	
	\lstinputlisting [language=Matlab, caption={
	Avalia pertinência da regras de \(D_1\) e \(D_3\) em função das distâncias
	medidas.}, label={lst:d1d3}] {fuzzy/get_D1_D3_Rule.m}

	\vspace{12pt}
	
	A função \texttt{get\_D2\_Rule} realiza o mesmo que o descrito anteriormente,
	mas agora com o sensor \(D_2\). A implementação desta função pode ser
	encontrada logo abaixo.
	
	\lstinputlisting [language=Matlab, caption={
	Avalia pertinência da regras de \(D_2\) em função das distâncias
	medidas.}, label={lst:d2}] {fuzzy/get_D2_Rule.m} 
	
	\vspace{12pt}
	
	A quarta função, \texttt{get\_Angle\_Rule}, implementa as regras descritas na
	tabela \ref{tab:consequentes}. Por ser extensa, a sua implementação foi
	colocada na seção \textbf{Anexos} ao fim deste documento no programa
	\ref{lst:angle_rule}. Em poucas palavras, esta função testa se cada uma das
	regras está ativa e, caso uma determinada regra esteja, atribui a ela o menor
	valor dentre as pertinências dos estados que a compõem. Por exemplo, se \(D_1\)
	é \textbf{P} com 50\%, \(D_2\) é \textbf{MG} com 100\% e \(D_3\) é \textbf{M}
	com 33\%, então \(\Delta \theta\) é \textbf{PN} com 33\%.
	
	\vspace{12pt}
	
	A quinta função, \texttt{get\_Angle}, é responsável pelo processo de
	\textit{defuzzyficação} e está representada pelo programa \ref{lst:angle}. Ela
	utiliza o método de centro de massa para determinar \(\Delta \theta\) que
	melhor convém dadas as regras ativas.
	
	\lstinputlisting [language=Matlab, caption={
	Avalia pertinência da regras de \(D_2\) em função das distâncias
	medidas.}, label={lst:angle}] {fuzzy/get_Angle.m} 
