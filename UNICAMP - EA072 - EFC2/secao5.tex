\section{Sistema de recomendação – Paradigma alternativo}

\subsection{Funcionamento do SCOAL}

Conforme explorado na seção anterior, o algoritmo \textit{k-NN} primeiramente
divide a matriz de dados em \textit{co-clusters} e, em seguida, realiza uma
predição baseada em um desses \textit{co-clusters} gerados. Tal algoritmo,
portanto, não leva em consideração os atributos dos filmes e dos usuários, isto
é, a predição é baseada exclusivamente nos \textit{ratings} dos usuários em uma
vizinhança. O algoritmo \textit{SCOAL}, por sua vez, utiliza ambas as
informações de vizinhança e atributos para predizer uma nota. A ideia do
\textit{SCOAL} é dividir toda a matriz em grupos, ou \textit{clusters}, de
maneira que cada um possa ser bem caracterizado por um único modelo preditivo. A
similaridade é dada, portanto, pela similiridade dos modelos preditivos e não
somente dos valores dos \textit{ratings}, como acontece no \textit{k-NN}. Além
disso, o \textit{SCOAL} realiza o agrupamento, ou \textit{co-clustering}, simultaneamente
com a obtenção dos respectivos modelos de classificação, a fim de melhorar a
designação dos dados aos \textit{clusters} e a precisão dos modelos.

\subsection{A base de dados \textit{MovieLens}}

Os dados utilizados nesta seção foram disponibilizados pelo grupo
\textit{GroupLens} no mês de Abril de 1998. São 100.000 \textit{ratings},
compreendidas no intervalo \([1,5]\), dadas por 943 usuários em 1682 filmes e
coletadas durante 7 meses, indo de 19 de setembro de 1997 até 22 de abril de
1998. É importante dizer que esses dados foram filtrados, de forma que usuários
com menos de 20 \textit{ratings} ou que não possuíssem informações
demográficas completas foram eliminados da base. Essas informações demográficas
são compostas por idade, sexo, ocupação e endereço \textit{zip}. Filmes, além
de informações sobre título, data de lançamento e \textit{url} do IMDb,
podem possuir diversos gêneros (ação, comédia, aventura etc).

\vspace{12pt}

Esses dados foram disponibilizados em alguns arquivos, cujos os principais são:

\begin{itemize}
  \item \texttt{u.data}: arquivo contendo as classificações dos usuários. Cada
  linha, além de possuir a respectiva \textit{rating}, associa um \textit{id} de
  usuário a um \textit{id} de filme. Adicionalmente, há um outro atribuot que
  descreve a data que a respectiva classificação foi dada.

  \item \texttt{u.item}: possui as informações apresentadas acima sobre os
  filmes.
  
  \item \texttt{u.user}: possui as informações apresentadas acima sobre os
  usuários.
  
\end{itemize}

\subsection{Execução do \textit{toolbox} fornecido}