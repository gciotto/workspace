\section{Conclusões}

Este exercício de fixação de conceitos mostrou todo o poder e abrangência das
técnicas de inteligência artificial, visto que diversos problemas de áreas
completamente distintas foram abordados e resolvidos. A primeira aplicação
consistiu na determinação dos parâmetros \(k_i\), \(k_d\) e \(k_p\) de um
controlador através de um algoritmo evolutivo. A determinação desses parâmetros
é complexa e exige um amplo conhecimento dos conceitos de controle e automação.
Sendo assim, o interesse da utilização de um algoritmo de IA é justamente propor
um método que possa contornar esse elevado grau de complexidade sem conhecer
intrinsecamente o problema, ainda que a solução encontrada não seja a ótima.
Destaca-se, entretanto, que as soluções encontradas foram muito satisfatórias
para os critérios selecionados, isto é, tempo de resposta e margem de fase.

\vspace{12pt}

Para a segunda aplicação, utilizamos o \textit{software} Eureqa para produzir
mapeamentos a partir de dados. Observou-se que, mesmo para dados contendo ruído,
o programa é capaz de produzir aproximações muito consistentes. Foi possível
igualmente manipular a ideia de equilíbrio de Pareto, à medida que várias
situações eram apresentadas com diversos graus de complexidade e precisão.

\vspace{12pt}

A terceira aplicação envolveu os conceitos de controle nebuloso e redes neurais
MLP. O controle do robô foi realizado por ambos os modelos. Foi possível
visualizar, portanto, que, apesar desses métodos serem baseados em princípios
completamente distintos, ambos conseguem resolver o problema. Uma das
características mais importantes dos algoritmos de inteligência artifial é
justamente a sua generalidade em resolver problemas de diferentes áreas.

\vspace{12pt}

Enfim, para os quarto e quintos exercícios, abordou-se o problema de
recomendação de notas para os usuários. Para isso, usamos a base
\textit{MovieLens} que contém uma grande quantidade de \textit{ratings}. O
primeiro método explorado foi o \textit{k-NN}, que não leva em consideração os
atributos de usuários e filmes na sua predição. Este algoritmo só considera os
\(k\) vizinhos mais próximos de um usuário determinado. Exploramos também o
algoritmo SCOAL, mais robusto que o primeiro, que utiliza os atributos
discutidos anteriormente e divide a matriz de dados em \textit{co-clusters},
para os quais modelos preditivos são calculados.
