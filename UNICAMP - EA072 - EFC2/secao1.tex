\section{Síntese de controle PID}

\begin {enumerate}
  \item Antes de definirmos os operadores de mutação e \textit{crossover}, é
  necessário diferenciarmos estes dois conceitos. Embora ambos ocorram em
  organismos biológicos e sejam responsáveis pela alteração do genótipo de um
  indivíduo, estes mecanismos ocorrem em diferentes etapas da vida do respectivo
  organismo. O termo \textit{mutação} é utilizado para descrever o processo no
  qual um alelo de um gene é \textbf{aleatoriamente} substituído ou modificado
  por outro. Em termos matemáticos, sejam os índices \(r, \ldots, u\) as
  posições que sofrerão mutação e que foram determinadas aleatoriamente. Cada
  posição possui um probabalidade \(p_m\) de ser submetida a este processo. Ao
  final da mutação, os alelos referentes a estes índices serão modificados, no
  caso de uma codificação em ponto flutuante, segundo uma distribuição
  \textit{uniforme} ou \textit{não uniforme}. No caso da \textit{não uniforme},
  insere-se uma perturbação com distribuição \textit{normal} com média nula na
  posição escolhida e desvio-padrão decrescente ao longo das gerações, o que
  garante consequentemente um refinamento da solução à medida que nos
  aproximamos de resultados ótimos. 
  
  O processo de \textit{crossover}, por sua vez, trata-se de um mecanismo de
  recombinação genética de dois ou mais cromossomos. Para a codificação de
  algoritmos evolutivos em que os cromossomos possuem valores em ponto
  flutuante, destacam-se duas técnicas: o \textit{crossover} aritmético e o
  uniforme. Para o primeiro, a partir dos cromossomos \(\textbf{x}\) e \(\textbf{y}\),
  obtém-se uma combinção convexa dos valores dos genes de \(\textbf{x}\) e
  \(\textbf{y}\), isto é, o cromossomo \(a\textbf{x} + (1-a)\textbf{y}\), em que
  \(a \in [0, +1]\). Para o segundo, os genes dos cromossomos pais são
  escolhidos com igual probabilidade para formar um cromossomo filho. Por
  exemplo, se a probabilidade vale 0.5, então espera-se que o cromossomo
  resultante seja composto por metade dos genes de \textbf{x} e metade de
  \textbf{y}.
  
  Outra etapa importante para os algoritmos evolutivos é a seleção dos
  indivíduos mais ``adaptados'' ao problema (que possuem maior função de
  \textit{fitness}). Este processo deve garantir que os melhores indivíduos
  persistam, mas não deve ser extremamente radical a fim de garantir uma
  diversidade na população. Um exemplo de um operador de seleção é a técnica de
  \textit{seleção por torneio}. Para selecionar \(N\) indivíduos, realizam-se
  \(N\) torneios com \(p\) participantes, escolhidos aleatoriamente. Quanto mais
  alto é \(p\), maior é a pressão seletiva, isto é, para um indivíduo ruim ser
  escolhido ao menos em um torneio, é necessário que ele compita com \(p-1\)
  indivíduos piores que ele (para \(p\) grande, a probabilidade deste fato
  ocorrer é muito pequena). Cada torneio é vencido pelo indivíduo que apresenta maior
  \textit{fitness}. No caso deste exercício, será realizado apenas um torneio,
  que é composto por 3 indivíduos.

  \item As constantes apresentadas na tabela \ref{tab:pid_constantes} a seguir
  foram definidas no arquivo \texttt{prog\_PID.m}, disponibilizado pelo
  professor.
  
  \begin{table}[h]
	    \centering
		\caption{\label{tab:pid_constantes} Constantes definidas em
		\texttt{prog\_PID.m}}
		\begin{tabular}{|c | c |}
			\hline
			\textbf{Atributos} & \textbf{Valor} \\	\hhline{|=|=|}
			Tamanho da População & 100 \\ \hline 
			Número máximo de Gerações & 50 \\ \hline 
			Taxa de Mutação & 0.4 \\ \hline 			
			Taxa de Crossover & 0.8 \\ \hline 			
		\end{tabular}	    
    \end{table} 
    
    Observa-se taxas de mutação e \textit{crossover} relativamente altas: para a
    primeira, um alelo tem probalidade de 40\% de ser mutado, isto é, ele possui
    aproximadamente a metade das chances de ser modificado. Em relação ao
    tamanho da população e o número de gerações , 100 e 50 são
    números suficientemente grandes para garantir, respectivamente,	 diversidade
    entre os indivíduos e uma boa qualidade no resultado visto a quantidade de
    recombinações entre os cromossomos que serão realizadas.
    
    A população inicial é obtida pelo comando \texttt{pop =
    5*rand(tam\_pop,3);}, em que \texttt{rand} é uma função do MATLAB que gera
    números aleatórios segundo uma distribuição uniforme e \texttt{tam\_pop}
    vale 100. Será criada, portanto, uma matriz de 100 linhas, cada uma
    representando um indivíduo, e 3 colunas, uma para cada constante a ser
    determinada \(k_p\), \(k_d\) e \(k_i\).
    
    Finalmente, cada operador de \textit{crossover} é escolhido com 50\% de
    probabilidade. Espera-se portanto que o operador aritmétido seja escolhido
    em metade das oportunidades e o uniforme, também.
    
    \item 
\end{enumerate}