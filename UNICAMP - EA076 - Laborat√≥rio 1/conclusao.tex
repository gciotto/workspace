\section{Conclusão}

Trabalhamos neste experimento alguns conceitos importantes envolvendo a
programação de sistemas embarcados. O primeiro foi a necessidade de consulta
frequente aos manuais fornecidos pelos fabricantes, a fim de obtermos os
limites elétricos dos componentes e, assim, determinar os circuitos que
ligaremos ao microcontrolador. O segundo aspecto foi a possibilidade de utilizar
um mesmo pino em diversas funções diferentes, seja como entrada/saída ou
comunicação serial, por exemplo. Neste quesito, aprendemos também como
identificar a localização  de um determinado pino na placa.

\vspace{12pt}

Exploramos igualmente o ambiente de desenvolvimento \textit{Code Warrior}, que
oferece muitos recursos na programação de \textit{firmwares}, tais como a
geração automática de código e interfaces de configuração dos pinos. Na
disciplina EA871, por exemplo, somos obrigados a configurar todos os
registradores individualmente, o que dificulta em algumas ocasiões o
\textit{debugging} do programa. Em oposição, em EA076, permitimos que o
\textit{Code Warrior} se encarregue destes detalhes.

\vspace{12pt}

A construção de diagramas de blocos evidenciou características importantes no
desenvolvimento, como a re-utilização de código e portabilidade de componentes
(isto é, um mesmo código pode funcionar com diversos \textit{hardwares}
distintos). Em escala industriais, tais aspectos são traduzidos em economia de
recursos e em uma maior facilidade de implementação tanto de \textit{software}
quanto de \textit{hardware}.
