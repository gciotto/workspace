\section {Introdução}

O principal propósito deste primeiro experimento foi explorar as ferramentas de
disponibilizadas pelo \textit{software} \textit{Code Warrior}, uma interface
gráfica baseada na \textit{IDE Eclipse}, para o desenvolvimento de
\textit{firmwares} compatíveis com microcontrolador \textit{Freescale KL25}.
Sendo assim, utilizamos tal plataforma para configurar alguns pinos do
microcontrolador como entrada ou saída, dependendo da montagem desejada. Foram
realizadas, no total, três montagens distintas.

\vspace{12pt}

Na primeira, o objetivo foi construir um programa capaz de piscar o \textit{LED}
vermelho, presente no \textit{kit}, em uma frequência de 1Hz. Para tal, foi
necessário consultar os manuais do microcontrolador [\ref{bib:manual}] para
entender como a comunicação entre ele e o diodo poderia ser realizada e como gerar interrupções
em instantes periódicos de tempo. Em seguida, levando em conta que o
\textit{kit} possui ainda outros dois \textit{LEDs}, extendemos o programa para piscar
também o \textit{LED} verde, a uma frequência de 2Hz.

\vspace{12pt}

A segunda tarefa consistiu na introdução de um \textit{LED} externo, de
modelo \textit{TLDR490} fabricado pela empresa \textit{Vishay} e cuja
documentação é disponibilizada no site do curso [\ref{bib:led}]. Para
realizá-la, foi preciso escolher um pino disponível do microcontrolador para funcionar como
\textit{output} e calcular a resistência necessária para limitar a corrente a
fim de não danificar nenhum dos dois componentes.

\vspace{12pt}

Enfim, a última tarefa envolveu o uso de um \textit{push-button} para fornecer
ao usuário a possibilidade de controlar o funcionamento dos \textit{LEDs}. Para
isso, tivemos de configurar um pino como entrada, assim como escolher
uma resistência para limitar a corrente que passa pelo circuito. Essa
resistência é chamada de \textit{pull} em referência à função que desempenha:
pode induzir um estado padrão alto (\textit{pull up}) ou baixo (\textit{pull
down}) na saída do circuito, que, nosso caso, é o pino que foi configurado como
\textit{input}.

\vspace{12pt}

As próximas seções visam a análise dessas montagens.