\section{Conclusões Finais} 
 
A partir dos exemplos práticos desenvolvidos nessa lista, foi possível explorar
as principais vantagens, desvantagens e recursos de alguns conceitos abordados
durante a aula. Primeiramente, para a seleção de variáveis e filtros
constatou-se que, ao contrário destes últimos, nem sempre as variáveis mais
correlatas com a saída são escolhidas para participar do modelo.
Adicionalmente, apesar dos métodos \textit{wrappers} apresentarem uma
complexidade computacional maior que os filtros, eles selecionam as variáveis
de modo que o máximo de informação fique contido no modelo, isto é, o erro
calculado junto ao modelo linear seja mínimo. Observou-se que os resultados
para o \textit{backward elimination} e \textit{forward selection} foram 
bastante similares, diferenciando-se somente em um número bem pequeno de
variáveis que foram selecionadas em um método, mas não no outro. Essa afirmação
reforça a ideia que ambos os modelos selecionam as melhores variáveis para
compor o modelo. 

\vspace{12pt}

Em relação aos preditores implementados, nos deparamos com o compromisso
"recursos computacionais \(\times\) resultados", isto é, à medida que
adicionamos a um determinado método algum "grau de liberdade" (como número de
neurônios ou de iterações, por exemplo), tendemos a encontrar uma solução
apresentando um erro menor. Isso foi observado principalmente no duelo entre os
preditores lineares e as redes neurais MLP. Para estas últimas, realizou-se
vários treinamentos distintos (da ordem de 1800 treinamentos - rever seção
\ref{sec:mlp}), mas, ao fim, encontra-se um resultado melhor em relação a ambos
os preditores lineares (regularizado e não regularizado). Um resultado que foi
constatato e que não era esperado é o fato de que as ELMs treinadas na seção
\ref{sec:elm}  apresentarem performance inferior até mesmo em relação aos
preditores lineares. Neste caso, foram treinadas várias ELMs com grandes
variações no número de neurônios nas suas camadas intermediárias e no parâmetro
regularizador \(c\). Mesmo com essa quantidade de processamento, os resultados
encontrados foram insuficientes, se comparados às MLPs e preditores lineares.

\vspace{12pt} 

Enfim, de maneira geral, este exercício foi bastante abrangente, permitindo aos
alunos desenvolver suas habilidades ligadas à programação de redes neurais em
MATLAB.
